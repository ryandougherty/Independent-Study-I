\section{Counting Complexity}

We have defined the $\P$ class as the set of languages that are solvable on a deterministic TM in polynomial time. However, we did not actually describe the solution(s) - for example, we only described whether a given 3SAT formula is satisfiable, and not how many satisfying solutions it has. Note that describing solutions is not a decision problem. However, we will make the necessary modifications:

\newcommand{\SharpP}{\lang{\#P}}
\begin{definition}
A function $f \in \SharpP$ if there is a polynomial $p$ and a poly-time TM $M$ such that for all $x$, $f(x) = |\left \{y \in \{0, 1\}^{p(|x|)} \colon M(x, y) = 1\right \}|$.
\end{definition}

We can see that instead of determining membership, $f$ counts the number of solutions such that the TM accepts. Clearly, the problems in $\SharpP$ are ``harder" than those in $\P$. But what if we want to have a decision version on $\SharpP$? We can do so with $\PP$:

\begin{definition}
We have $L \in \PP$ if there is a poly-time TM $M$ and a polynomial $p$ such that for all $x$, $x \in L$ if and only if $|\left\{u \in \{0, 1\}^{p(|x|)} \colon M(x, u) = 1\right\} | \ge 2^{p(|x|)-1}$.
\end{definition}
We can see that $\PP$ is basically a ``majority selector," in that $x \in L$ if more than half of all certificates have the TM accept. For the next theorem, we define $\FP$ as the set of functions computable by a deterministic poly-time TM.

\begin{theorem}
$\PP = \P$ if and only if $\SharpP = \FP$. 
\end{theorem}

\begin{proof}
It is easy to see that $\SharpP = \FP$ implies $\PP = \P$: given input $x$, just count the number of $u$ such that $M(x, u) = 1$, and compare. For the other direction
\end{proof}
\Comment{Complete this proof}

\subsection{$\SharpP$-completeness}
\begin{definition}
Define $\FP^{f}$ for a function $f$ to be the set of functions computable by poly-time TMs with oracle access to $f$ (same notation as for problems). We have $f$ to be $\SharpP$-complete if:
\begin{itemize}
\item $f \in \SharpP$
\item For all $g \in \SharpP$, $g \in \FP^{f}$.
\end{itemize}
\end{definition}

\newcommand{\SharpSAT}{\lang{\#\SAT}}
\begin{theorem}
$\SharpSAT$ is $\SharpP$-complete.
\end{theorem}

\begin{proof}
The reasoning is that the reduction for $\SAT$ is \emph{parsimonious} (i.e., is bijective and preserves counts). 
\end{proof}

\newcommand{\SharpPSPACE}{\lang{\#\PSPACE}}
\subsection{$\SharpPSPACE$}
We have talked about $\SharpP$ - what about $\SharpPSPACE$? We define it below:
\begin{definition}
A function $f \in \SharpPSPACE$ if there is a poly-time TM $M$ such that $f(x)$ outputs the number of possible $x_1, \cdots, x_k$ such that $Q_1x_1 \cdots Q_kx_k [M(x_1, \cdots, x_k) = 1]$. 
\end{definition}
\newcommand{\SharpQBF}{\lang{\#QBF}}
We also define $\SharpPSPACE$-completeness the same way we have $\SharpP$-completeness. We can easily see that $\SharpQBF \in \SharpPSPACE$, where $\SharpQBF$ is the same as the $\lang{QBF}$ problem but we count the number of solutions. We will show that $\SharpQBF = \SharpSAT$, thereby giving the result that:
\begin{theorem}
$\SharpP = \SharpPSPACE$.
\end{theorem}

\begin{proof}
The way we will prove this is showing $\SharpQBF$ is $\SharpPSPACE$-complete, and $\SharpQBF = \SharpSAT$.

We clearly have $\SharpQBF \in \SharpPSPACE$ (there is a poly-space TM $M$ that for any boolean formula $F$, and quantification of variables, $M$ accepts on the quantification if and only if the quantification for the formula is true).

\par Let $f \in \SharpPSPACE$. There is a poly-space TM $M$ and polynomial $p$ such that $f(x)$ is the number of $x_1, \cdots x_k$ has $Q_1x_1\cdots Q_{p(|x|)}x_{p(|x|)} M(x, x_1, \cdots, x_{p(|x|)})$ accepts.

\par Let $F_x(x_1, \cdots x_{p(|x|)})$ be the equivalent formula of $M$ (this is of poly-size). Therefore, $f(x)$ is the number of quantifications such that $F_x$ is true, which is equal to counting the solutions of the formula (i.e., $\SharpQBF(F_x)$). Therefore, $f \le_{\SharpP} \SharpQBF$. 

\par Now we show $\SharpQBF = \SharpSAT$. Let $F$ be a boolean formula on $n$ variables. We prove by induction on $n$. For $n=0$, the two formulas with no variables are the true formula $t$ and the false one $f$. $\SharpSAT(t) = \SharpQBF(t) = 1$ (there are no quantifications), and $\SharpSAT(f) = \SharpQBF(f) = 0$. 

\par We assume that $\SharpSAT(m) = \SharpQBF(m)$ for all formulas $m$ with $n-1$ variables. Let $F'$ be a formula on $n$ variables, and let $F'_t, F'_f$ be the same formula but with the last variable made true and false, respectively. We note the following:
\begin{itemize}
\item $\SharpSAT(F'_t \vee F'_f) = \SharpSAT(F'_t) + \SharpSAT(F'_f) - \SharpSAT(F'_t \wedge F'_f)$.
\item Let $Q^kx = Q_1x_1 \cdots Q_kx_k$ be the first $k$ quantified variables of $F'$. Then $Q^{n-1}x\forall x_n F'$ is true if and only if $Q^{n-1}x(F'_t \wedge F'_f)$ is true, and $Q^{n-1}x\exists x_n F'$ is true if and only if $Q^{n-1}x(F'_t \vee F'_f)$ is true. 
\end{itemize}
We also see that $\SharpQBF(F')$ is the number of quantifications of $F'$ that make it true with the $n$th variable as $\exists$ + the same but the $n$th variable is $\forall$. Therefore, $\SharpQBF(F') = \SharpQBF(F'_t \wedge F'_f) + \SharpQBF(F'_t \vee F'_f)$.

\par We assume by the induction hypothesis that this is equivalent to $\SharpSAT(F'_t \wedge F'_f) + \SharpSAT(F'_t \vee F'_f)$. Also, by the first observation, this is equivalent to $\SharpSAT(F'_t) + \SharpSAT(F'_f)$, which is equal to $\SharpSAT(F')$. 

\par We have $\SharpP = \SharpPSPACE$ because of our definition of $\SharpP$-completeness. 

%We introduce the ``circuit value problem" (CVP), which is: given a boolean formula $\phi$ and an assignment $x$ to its variables, is $\phi$ true with this assignment? It is much easier than solving $\SAT$, and is in $\P$. 
%
%\par We show that $\SharpQBF \in \SharpPSPACE$: let $M$ be a TM that decides CVP, and $Q_1x_1 \cdots Q_kx_k [\phi]$ be an arbitrary QBF. Therefore, the number of settings to the $x_i$ such that $M(\phi, x)$ accepts is precisely the number of $x_i$ settings that make $\phi$ true. Therefore, $\SharpQBF \in \SharpPSPACE$.
%
%\par Let $f \in \SharpPSPACE$. Therefore, there is a poly-time TM $M$ such that $f(x)$ is the number of $Q_ix_i$ settings that $M(x) = 1$. Let $L$ be the set of $x$ such that $M(x) = 1$. We can see that $L \in \P$ (just run $M$ on the input). Let $g$ be the poly-time reduction from $L$ to CVP (i.e., $M(x) = 1$ if and only if $g(x)$ is a formula with a satisfying truth assignment). 
%
%\par Therefore, $f(x)$ is the number of $Q_ig(x_i)$ settings that has the 
\end{proof}

\subsection{Toda's Theorem}
How powerful is $\SharpP$? It has been hard to determine until 1989, with one amazing result by Toda:
\begin{theorem}
$\PH \subseteq \P^{\#\SAT}$. 
\end{theorem}
This means that with any $\SharpP$-complete problem, we can solve any problem in $\PH$ (i.e., a subexponential algorithm). We build the necessary ingredients for the proof:

\newcommand{\Parity}{\oplus}
\newcommand{\ParityP}{\lang{\Parity\P}}
\begin{definition}
We have $\L \in \ParityP$ (``parity-$\P$") if there is a poly-time \emph{nondeterministic} TM $M$ such that $x \in L$ if and only if there are an odd number of accepting paths of $M$ on $x$. 
\end{definition}
\newcommand{\ParitySAT}{\Parity\SAT}
Also, we need to know about the $\Parity$ quantifier as well as $\ParitySAT$:
\begin{definition}
For a boolean formula $\phi$, define $\Parity\phi(x)$ to be true if the number of $x$'s that make $\phi$ true is odd. Define $\ParitySAT$ to be the language of $\Parity\phi(x)$ where $\phi$ is an unquantified boolean formula in any form.
\end{definition}
We can reason that $\ParityP$ corresponds to the least significant bit of problems in $\SharpP$, so it doesn't seem very powerful. However, in the first part of the proof, Toda shows a randomized reduction from $\PH$ to $\ParitySAT$. 

\begin{theorem}
\label{thm:parity_sat_randomized_reduction}
Let $c \in \mathbb{N}$. There is a probabilistic poly-time algorithm $A$ with parameter $m$ any quantified boolean formula $\phi$ with $|\phi| = n$ with $c$ alternations and runs in poly-time such that:
\begin{itemize}
\item $\phi$ is true implies $\Pr[A(\phi) \in \ParitySAT] \ge 1-2^{-m}$
\item $\phi$ is false implies $\Pr[A(\phi) \in \ParitySAT] \le 2^{-m}$.
\end{itemize}
\end{theorem}
We prove this theorem later, but the basic idea is to use this in the randomized reduction. But first, we need to be able to reduce $\NP$ to $\ParitySAT$. We rely on earlier work by Valiant and Vazirani:
\newcommand{\USAT}{\lang{USAT}}
\begin{theorem}
\label{thm:valiant_vazirani}
Let $\USAT$ be the set of boolean formulas that has a single satisfying assignment. There is a probabilistic poly-time algorithm $f$ such that for any $n$-variable boolean formula $\phi$:
\begin{itemize}
\item $\phi \in \SAT$ implies $\Pr[f(\phi) \in \USAT] \ge \frac{1}{8n}$,
\item $\phi \notin \SAT$ implies $\Pr[f(\phi) \in \SAT] = 0$.
\end{itemize}
\end{theorem}
In order to prove \Cref{thm:valiant_vazirani}, we need to prove the following lemma:
\begin{lemma}
Let $H_{n,k}$ be a set of pairwise independent hash functions from $\{0,1\}^n$ to $\{0,1\}^k$. Let $S \subseteq \{0, 1\}^n$ such that $2^{k-2} \le |S| \le 2^{k-1}$. Then for all $h \in H_{n, k}$, the probability of a unique $x \in S$ satisfying $h(x) = 0^k$ is at least $\frac{1}{8}$.
\end{lemma}
\begin{proof}
For all $x \in S$, denote $p$ be the probability that $h(x) = 0^k$. For any other $x'$, the probability of $h(x) = h(x')$ is $p^2$. Let $N$ be the random variable that is the number of $x \in S$ with $h(x) = 0^k$. We have $\mathbb{E}[N] = |S|p$, which is either $\frac{1}{4}$ or $\frac{1}{2}$. Therefore, we have:
\[
\Pr[N \ge 1] \ge \sum_{x \in S} \Pr[h(x) = 0^k] - \sum_{x < x'} \Pr[h(x) = h(x') = 0^k]
\]
which is just equal to $|S|p - {|S| \choose 2}p^2$. By the union bound, we have $\Pr[N \ge 2] \le {|S| \choose 2}p^2$. Therefore, subtracting these two quantities has $\Pr[N=1] = \Pr[N \ge 1] - \Pr[N \ge 2] \ge |S|p - |S|^2p^2 \ge \frac{1}{8}$.
\end{proof}
Now we prove \Cref{thm:valiant_vazirani}:
\begin{proof}
For any $\phi$ on $n$ variables, choose $k$ at random from $\{2, \cdots, n+1\}$ and a random hash function $h$. Consider $\exists_{x} \phi(x) \wedge h(x) = 0^k$. If $\phi \notin \SAT$, then this statement is false. If $\phi \in \SAT$, then with probability $\frac{1}{8n}$ there is a unique assignment $x$ satisfying the statement. If $S$ is the set of $\phi$'s satisfiable variable settings, then with probability $\frac{1}{n}$, $2^{k-2} \le |S| \le 2^{k-1}$, based on the previous lemma that with probability $\frac{1}{8}$ there is a unique $x$ that $\phi(x) \wedge h(x) = 0^k$ is true.

\par So how do we construct the poly-time transformation? We create a formula $\tau$ on $x \in \{0, 1\}^n$ and $y \in \{0, 1\}^{p(n)}$ for a polynomial $p$, with the property that $h(x) = 0$ if and only if there is a unique $y$ with $\tau(x, y) = 1$. The output formula is $\phi(x) \wedge \tau(x, y)$. 
\end{proof}

Note that as a result, we have that there is a probabilistic poly-time algorithm $A$ such that:
\begin{itemize}
\item $\phi \in \SAT$ implies that $\Pr[A(\phi) \in \ParitySAT] \ge \frac{1}{8n}$,
\item $\phi \notin \SAT$ implies that $\Pr[A(\phi) \in \ParitySAT] = 0$
\end{itemize}

We need more notation. Let $\#(\phi)$ be the number of satisfying assignments of $\phi$. We can construct a formula on two different formulas over different variables (say, the functions are $\phi, \psi$; variables are $x_i \in \{0, 1\}^n, y_i \in \{0, 1\}^m$) with $n+m$ variables called $\phi \cdot \psi$, and a $\max\{n, m\}+1$-variable formula $\phi+\psi$ with the properties:
\begin{itemize}
\item $\#(\phi \cdot \psi) = \#(\phi)\#(\psi)$,
\item $\#(\phi+\psi) = \#(\phi)+\#(\psi)$.
\end{itemize}
The first formula is constructed by an $\wedge$ of the 2 formulas, and the second by mapping $(\phi+\psi)(z)$ to $((z_0=0) \wedge \phi(z_1, \cdots, z_n)) \vee ((z_0=1) \wedge (z_{m+1} = 0) \wedge \cdots \wedge (z_n = 0) \wedge \psi(z_1, \cdots, z_m))$ for some $m < n$. Also, we say $\phi+1$ is the same as $\phi+\psi$ but $\psi$ has a single satisfying assignment. 

\par Now we describe the $\Parity$ operator; we can see that:
\begin{itemize}
\item $\Parity_x \phi(x) \wedge \Parity_y \psi(y)$ is true if and only if $\Parity_{x,y}(\phi \cdot \psi)(x, y)$ is true. The reason for this is that any polynomially many number of $\wedge, \vee$ of $\ParitySAT$ instances can be converted into a single equivalent $\ParitySAT$ instance.
\item $\neg \Parity_x \phi(x)$ is true if and only if $\Parity_{x,z}(\phi+1)(x, z)$ is true. The reason for this is that $\ParityP$ is closed under complement (we can construct a formula in poly-time that is the complement of the original).
\item $\Parity_x \phi(x) \vee \Parity_y \psi(y)$ is true if and only if $\Parity_{x,y,z} ((\phi+1)\cdot (\psi+1) + 1)(x,y,z)$ is true. The reasoning is the same as the first.
\end{itemize}

\par Now we prove \Cref{thm:parity_sat_randomized_reduction}, with the restriction that the formula only has the $\exists$ quantifier:
\begin{proof}
We already proved the $c=1$ case, which was $\NP$. Let $\phi$ be a boolean formula. We run the above algorithm $r = O(mn)$ times, and the final formula is the $\vee$ of all of the produced formulas. If $\phi \in \SAT$, then the result is with probability $\ge 1-(1-\frac{1}{8n})^r = 1-2^{-m}$. If $\phi \notin \SAT$, then the produced formula is not true either. We apply the third observation about the $\Parity$ operator $r$ times to convert it into a single $\Parity$ formula. We can do this since the size increase is only a polynomial.
\end{proof}

\par Now we do the general case. However, we abstract the V-V lemma to work with arbitrary formulas:
\begin{lemma}
There is a probabilistic poly-time algorithm $f$ that, on input $1^n$, produces a formula $\tau(x, y)$ with $x$ consisting of $n$ boolean variables such that, for any boolean function $\beta$:
\begin{itemize}
\item $\exists_{x_1}$ such that $\beta(x_1)$ is true implies that $\Pr[\Parity_{x_1,y}\tau(x_1,y) \wedge \beta(x_1) = 1] \ge \frac{1}{8n}$,
\item $\neg\exists_{x_1}$ such that $\beta(x_1)$ is true implies that $\Pr[\Parity_{x_1,y}\tau(x_1,y) \wedge \beta(x_1) = 1] = 0$.
\end{itemize}
\end{lemma}
With this, we prove the general case:
\begin{proof}
Let $\phi$ have $c$ alternations. Since $\ParityP$ (and of course, $\ParitySAT$) is closed under complement, we can assume the first quantifier is $\exists$. Therefore, $\phi = \exists_{x_1}\psi(x_1)$ with $\psi$ a quantified formula with $\le c-1$ alternations, and the variables in $x_1$ are free.

\par Suppose $x_1$ has $n$ variables. By induction, there is a randomized reduction that, for each value of $x_1$, it produces an equivalent (to $\phi(x_1)$) $\ParitySAT$ formula $\beta$ such that $\beta(x_1) = \Parity_z \rho(z, x_1)$ with probability $\ge 1-2^{-m-2}$. 

\par We run the abstract V-V lemma $r = O(mn)$ times, on independently random bits. Let $\{\tau_i (x_1, y)\}_{i=1}^r$ be the produced formulas.

\par Consider $\alpha = \bigvee_{j=1}^{r} (\Parity_{x_1, y} \tau_j(x_1, y) \wedge \beta(x_1))$. If $\exists_{x_1}\beta(x_1)$ is true, then the probability that $\alpha$ is true is $\ge 1-(1-\frac{1}{8n})^r = 1-2^{-O(m)}$. Also, if $\exists_{x_1}\beta(x_1)$ is false, then this probability is 0. 

\par To finish, we can have errors in converting $\phi(x_1)$ to an equivalent $\ParitySAT$ instance (with error $2^{-m-2}$), and when we replace $\exists_{x_1}$ in constructing $\alpha$, also with the same error. We have that the overall error is $2^{-m-1} \le 2^{-m}$, so we are done.
\end{proof}

\subsection*{Derandomization}
We now make the reduction deterministic. 
\begin{lemma}
There is a deterministic poly-time reduction $f$ that for any boolean formula $\alpha$, there is a boolean formula $\beta = T(\alpha, 1^n)$ such that:
\begin{itemize}
\item $\alpha \in \ParitySAT$ implies that $\#(\beta) = -1 (\mod 2^{n+1})$
\item $\alpha \notin \ParitySAT$ implies that $\#(\beta) = 0 (\mod 2^{n+1})$.
\end{itemize}
\end{lemma}
\begin{proof}
Note the constructions of $\#(\phi\cdot\tau), \#(\phi+\tau)$ from earlier. The resulting formulas are only a constant larger than either of $\phi, \tau$. Consider $4\tau^3 + 3\tau^4$, with $\tau^k = \tau \cdot (\tau^{k-1})$. We have that:
\begin{itemize}
\item $\#(\tau) = -1 (\mod 2^{2^i})$ implies that $(4\tau^3 + 3\tau^4) = -1 (\mod 2^{2^{i+1}})$.
\item $\#(\tau) = 0 (\mod 2^{2^i})$ implies that $(4\tau^3 + 3\tau^4) = 0 (\mod 2^{2^{i+1}})$.
\end{itemize}
Let $\psi_0 = \alpha$, and $\psi_{i+1} = 4\psi_i^3 + 3\psi_i^4$, and let $\beta = \psi_{\lceil \log(n+1)\rceil}$. We have that repeated use of the 2 implications above has that if $\#(\psi)$ is odd, then $\#(\beta) = -1 (\mod 2^{n+1})$. If $\#(\psi)$ is even, then $\#(\beta) = 0 (\mod 2^{n+1})$. We have $|\beta| = \exp(O(\log n))|\alpha|$, and so its size is polynomial in the input length.
\end{proof}

\subsection*{Proving Toda's Theorem}
\begin{proof}
Let $f$ be the randomized poly-time reduction above with $m=2$. Since it is randomized, we can think of it as a deterministic one with 2 inputs: a quantified formula $\psi$ and a random string $r$, with $R = |r|$. Let $T$ be the deterministic poly-time reduction above with $n = R+2$, which runs in $\poly(R, |f(\psi)|)$ time. 

\par Consider $T \cdot f$, and consider $\sum_{r \in \{0, 1\}^R} \#(T \cdot f(\psi, r)) (\mod 2^{n+1})$. If $\psi$ is true, then $\frac{3}{4}$ of the terms in the sum are $-1 (\mod 2^{n+1})$ and the others are $0 (\mod 2^{n+1})$. Therefore, the sum is between $-2^R$ and $-\lceil \frac{3}{4} \times 2^R \rceil$, both $(\mod 2^{n+1})$.

\par If $\psi$ is false, then by similar analysis we get that the sum is between $-\lceil \frac{1}{4} \times 2^R\rceil$ and 0 $(\mod 2^{n+1})$. 

\par Since $2^{n+1} > 2^{R+2}$, the ranges are disjoint. If we can evaluate the sum using a $\SharpSAT$ oracle, we can determine if $\psi$ is true or not. We can do this: let the variables of $T \cdot f(\phi, r)$ be $y$. We create a boolean formula $\Gamma(r, y, z)$ that is true for $(r, y, z)$ if and only if $y$ is a satisfying assignment for $T \cdot f(\phi, r)$. We can first compute $T \cdot(\phi, r)$ (with $\phi$ hard-coded into the circuit), and substitutes $y$ into this formula. The $z$ are the inner wires of the circuit. Since the circuit is deterministic, $z$ is uniquely determined by $y$ and $r$. 

\par Therefore, $\#(\Gamma(r,y,z)) (\mod 2^{n+1})$ is exactly the same as the sum above. Therefore, we just query the $\SharpSAT$ oracle for $\#(\Gamma)$.
\end{proof}
What is more unintuitive is that we only need 1 query for the $\SharpSAT$ oracle, instead of polynomially many. 