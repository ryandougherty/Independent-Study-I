\section{Review}
\label{sec:review}

This section highlights many of the key definitions and theorems studied in a first-year graduate (or advanced undergraduate) course in complexity theory. We assume the reader knows about finite automata (DFAs/NFAs), grammars (CFGs), and Turing machines (TMs), and their respective language classes.

\subsection{(Un)Decidability}

\begin{definition}
A TM is a \emph{decider} if it halts (accepts or rejects) on every input.
\end{definition}

\begin{definition}
A language $B$ is \emph{decidable} if there exists a decider $D$ such that $L(D) = B$. A language $C$ is \emph{undecidable} if $C$ is not decidable.
\end{definition}

\begin{theorem}
The following are decidable:
\begin{itemize}
\item $A_{DFA} = \{\langle M, w \rangle : M$ is a DFA that accepts $w$\}.
\item $E_{DFA} = \{\langle M \rangle : M$ is a DFA whose language is empty\}.
\item $ALL_{DFA} = \{\langle M \rangle : M$ is a DFA whose language is $\Sigma^*$\}.
\item $EQ_{DFA} = \{\langle M_1, M_2 \rangle : M_1$ and $M_2$ are DFAs and $L(M_1) = L(M_2)$\}.
\item $A_{CFG} = \{\langle G, w \rangle : G$ is a CFG that generates $w$\}.
\item $E_{CFG} = \{\langle G \rangle : L(G)$ is empty\}.
\end{itemize}
\end{theorem}

\begin{theorem}
The following are undecidable:
\begin{itemize}
\item $ALL_{CFG} = \{\langle G \rangle : G$ is a CFG and $L(G) = \Sigma^*$\}.
\item $EQ_{CFG} = \{\langle G_1, G_2 \rangle : G_1$ and $G_2$ are CFGs and $L(G_1) = L(G_2)$\}.
\item $A_{TM} = \{\langle M, w \rangle : M$ is a TM that accepts $w$\}.
\end{itemize}
\end{theorem}

\begin{theorem}
The class of decidable languages is closed under complement.
\end{theorem}

\begin{definition}
A language $B$ is \emph{Turing-recognizable} if there exists a TM that recognizes $B$. A language $C$ is co-Turing-recognizable (or co-recognizable) if it is the complement of some Turing-recognizable language. 
\end{definition}

\begin{theorem}
$A_{TM}$ is not co-recognizable.
\end{theorem}

\begin{theorem}
A language $B$ is decidable if and only if $B$ is recognizable and co-recognizable.
\end{theorem}

\begin{definition}
A function $f: \Sigma^* \rightarrow \Sigma^*$ is a \emph{computable function} if there exists a TM that, on input $w$, halts with $f(w)$ on its tape.
\end{definition}

\subsection{Reducibility}

\begin{definition}
A language $A$ is \emph{mapping-reducible} to language $B$, written $A \le_m B$, if there exists a computable function such that $w \in A$ if and only if $f(w) \in B$.
\end{definition}

\begin{theorem}
If $A \le_m B$ and $B$ is decidable, then $A$ is decidable; if $A$ is undecidable, then $B$ is undecidable.
\end{theorem}

\begin{corollary}
$HALT_{TM} = \{\langle M, w \rangle : M$ is a TM that halts on input $w$\} is undecidable.
\end{corollary}

\begin{definition}
A TM's language has a \emph{property} $P$ (a subset of all TM descriptions) such that whenever $M_1, M_2$ are TMs, and $L(M_1) = L(M_2)$, $\langle M_1 \rangle \in P$ if and only if $\langle M_2 \rangle \in P$. A property $P$ is \emph{nontrivial} if some TM has property $P$ and some other TM does not.
\end{definition}

\spnewtheorem{ricethm}[theorem]{(Rice's Theorem)}{\bfseries}{\itshape}
\begin{ricethm}
Deciding whether a TM has a nontrivial property $P$ of its language is undecidable.
\end{ricethm}