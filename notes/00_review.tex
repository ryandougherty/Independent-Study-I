\section{Review}
\label{sec:review}

This section highlights many of the key definitions and theorems studied in a first-year graduate (or advanced undergraduate) course in complexity theory. We assume the reader knows about finite automata (DFAs/NFAs), grammars (CFGs), and Turing machines (TMs), and their respective language classes.

\subsection{(Un)Decidability}

\begin{definition}
A TM is a \emph{decider} if it halts (accepts or rejects) on every input. A language $B$ is \emph{decidable} if there exists a decider $D$ such that $L(D) = B$. A language $C$ is \emph{undecidable} if $C$ is not decidable.
\end{definition}

\begin{theorem}
The following are decidable:
\begin{itemize}
\item $A_{DFA} = \{\langle M, w \rangle : M$ is a DFA that accepts $w$\}.
\item $E_{DFA} = \{\langle M \rangle : M$ is a DFA whose language is empty\}.
\item $ALL_{DFA} = \{\langle M \rangle : M$ is a DFA whose language is $\Sigma^*$\}.
\item $EQ_{DFA} = \{\langle M_1, M_2 \rangle : M_1$ and $M_2$ are DFAs and $L(M_1) = L(M_2)$\}.
\item $A_{CFG} = \{\langle G, w \rangle : G$ is a CFG that generates $w$\}.
\item $E_{CFG} = \{\langle G \rangle : L(G)$ is empty\}.
\end{itemize}
\end{theorem}

\begin{theorem}
The following are undecidable:
\begin{itemize}
\item $ALL_{CFG} = \{\langle G \rangle : G$ is a CFG and $L(G) = \Sigma^*$\}.
\item $EQ_{CFG} = \{\langle G_1, G_2 \rangle : G_1$ and $G_2$ are CFGs and $L(G_1) = L(G_2)$\}.
\item $A_{TM} = \{\langle M, w \rangle : M$ is a TM that accepts $w$\}.
\end{itemize}
\end{theorem}

\begin{theorem}
The class of decidable languages is closed under complement.
\end{theorem}

\begin{definition}
A language $B$ is \emph{Turing-recognizable} (or recognizable) if there exists a TM that recognizes $B$. A language $C$ is co-Turing-recognizable (or co-recognizable) if it is the complement of some Turing-recognizable language. 
\end{definition}

\begin{theorem}
$A_{TM}$ is not co-recognizable.
\end{theorem}

\begin{theorem}
A language $B$ is decidable if and only if $B$ is recognizable and co-recognizable.
\end{theorem}

\begin{definition}
A function $f: \Sigma^* \rightarrow \Sigma^*$ is a \emph{computable function} if there exists a TM that, on input $w$, halts with $f(w)$ on its tape.
\end{definition}

\subsection{Reducibility}

\begin{definition}
A language $A$ is \emph{mapping-reducible} to language $B$, written $A \le_m B$, if there exists a computable function $f$ such that $w \in A$ if and only if $f(w) \in B$.
\end{definition}

\begin{theorem}
If $A \le_m B$ and $B$ is decidable, then $A$ is decidable; if $A$ is undecidable, then $B$ is undecidable; if $B$ is recognizable, then $A$ is recognizable; if $A$ is not recognizable, then $B$ is not recognizable.
\end{theorem}

\begin{corollary}
$HALT_{TM} = \{\langle M, w \rangle : M$ is a TM that halts on input $w$\} is undecidable.
\end{corollary}

\begin{definition}
A TM's language has a \emph{property} $P$ (a subset of all TM descriptions) such that whenever $M_1, M_2$ are TMs, and $L(M_1) = L(M_2)$, $\langle M_1 \rangle \in P$ if and only if $\langle M_2 \rangle \in P$. A property $P$ is \emph{nontrivial} if some TM has property $P$ and some other TM does not.
\end{definition}

\begin{theorem}[Rice's Theorem]
Deciding whether a TM has a nontrivial property $P$ of its language is undecidable.
\end{theorem}

\begin{theorem}
$EQ_{TM} = \{\langle M_1, M_2 \rangle : M_1, M_2$ are TMs and $L(M_1) = L(M_2)$\} is undecidable; also, it is neither recognizable nor co-recognizable.
\end{theorem}

\begin{definition}
A \emph{configuration} of a TM on input $w = w_1\cdots w_n$ in state $q$ is $w_1\cdots w_{i-1}qw_i\cdots w_n$. A \emph{computation history} is a set of configurations delimited by an extra symbol \#: $\#C_1\#C_2\#\cdots\#C_{\ell}\#$, where $C_i$ logically yields $C_{i+1}$. An \emph{accepting computation history} is one such that $C_1$ is the start configuration, and $C_{\ell}$ is an accepting one.
\end{definition}

\begin{definition}
A \emph{linear bounded automaton} (LBA) is a TM that does not allow to move the tape head past the right end of the input.
\end{definition}

\begin{theorem}
$A_{LBA} = \{\langle M, w\rangle : M$ is an LBA that accepts $w$\} is decidable.
\end{theorem}

\begin{definition}
The \emph{Post Correspondence Problem} (PCP) is a puzzle, with a given set of tiles with nonempty ``top strings" and nonempty ``bottom strings." The objective is to list the tiles, repetitions allowed, such that the concatenation of the top strings of all the chosen tiles equals the same of the bottom strings.
\end{definition}

\begin{theorem}
PCP is undecidable.
\end{theorem}

\begin{theorem}[Recursion Theorem]
Let a TM $T$ compute a function $t: \Sigma^* \times \Sigma^* \rightarrow \Sigma^*$. Therefore, there exists a TM $R$ that computes a function $r: \Sigma^* \rightarrow \Sigma^*$, such that $r(w) = t(\langle R \rangle, w)$ for all $w$. In other words, every TM can obtain their own description.
\end{theorem}

\begin{definition}
A TM $M$ is \emph{minimal} if there does not exist a TM $N$ that has fewer states and $L(M) = L(N)$.
\end{definition}

\begin{theorem}
$MIN_{TM} = \{\langle M \rangle : M$ is a TM and is minimal\} is not recognizable.
\end{theorem}

\subsection{Logical Theories}

\begin{definition}
A formula over some operations is \emph{atomic} if $R_i$ over variables $x_1,\cdots,x_{\ell}$ is a relation of arity $\ell$. A formula $\phi$ is \emph{well-formed} if it is atomic, a formula formed from other atomic formulas using the operations, or of the form $\exists x[\phi_1]$ or $\forall x[\phi_1]$ where $\phi_1$ is a well-formed formula. A variable is \emph{bounded} if it is within the scope of a quantifier, and free otherwise. A well-formed formula with no free variables is a \emph{sentence} or a \emph{statement}. The \emph{universe} is the set of possible values for each variable, and the \emph{model} specifies the universe and relations used. The \emph{theory} of a model $M$, called ${\sc Th}(M)$, is the set of true statements. A formula in \emph{prenex normal form} is one that has all quantifiers appear first. 
\end{definition}

\begin{theorem}
$Th(\mathbb{N}, +)$ is decidable.
\end{theorem}

\begin{theorem}
$Th(\mathbb{N}, +, \times)$ is undecidable.
\end{theorem}

\begin{definition}
A \emph{formal proof} of a statement $\phi$ is a sequence of statements $S_1,\cdots,S_{\ell}$ where $S_{\ell} = \phi$, where each $S_i$ follows logically from preceding statements and \emph{axioms} (statements not requiring a proof). If all provable statements are true, then the system is \emph{sound}; if all true statements are provable, then the system is \emph{complete}.
\end{definition}

\begin{theorem}
$Provable(\mathbb{N}, +, \times) = \{$set of statements in $(\mathbb{N}, +, \times)$ that have proofs\} is recognizable.
\end{theorem}

\begin{theorem}
There exists a true, but unprovable statement in $Th(\mathbb{N}, +, \times)$.
\end{theorem}

\subsection{Oracle TMs}

\begin{definition}
An \emph{oracle TM} $M$ is a TM with an ``oracle tape" that, when the TM writes a string onto this tape, invokes the oracle (of a language $L$) and decides membership of the written string in $L$ in zero time, and returns a yes or no answer (written as $M^L$). A language $A$ is \emph{decidable relative to} a language $B$--$A \le_T B$--if there is a TM $M^B$ that decides $A$. $A$ is \emph{Turing-reducible} to $B$ if and only if $A \le_T B$.
\end{definition}

\begin{theorem}
$E_{TM} \le_T A_{TM}$.
\end{theorem}

\begin{theorem}
$A'_{TM} = \{\langle M, w\rangle : M$ is a TM with an oracle for $A_{TM}$ and $M$ accepts $w$\} is undecidable relative to $A_{TM}$.
\end{theorem}

\begin{definition}
The \emph{minimal description} of a string $x$ ($d(x)$)is the shortest string $\langle M, w\rangle$ where TM $M$, on input $w$, halts with $x$ on the tape. The \emph{descriptive complexity} of $x$ ($K(x)$) is $|d(x)|$
\end{definition}

\begin{theorem}
$K(x) \le |x| + c$ for a constant $c$.
\end{theorem}

\begin{theorem}
$K(xx) \le |x| + d$ for a constant $d$. 
\end{theorem}

\begin{theorem}
$K(xy) \le 2\log_2(K(x)) + K(x) + K(y) + e$ for a constant $e$.
\end{theorem}

\begin{definition}
A string $x$ is \emph{incompressible} if $K(x) \ge |x|$.
\end{definition}

\begin{theorem}
At least half of all strings of length $\le n$ are incompressible.
\end{theorem}

\begin{theorem}
$K(x)$ is not computable.
\end{theorem}

\begin{theorem}
No infinite subset of the set of incompressible strings is recognizable.
\end{theorem}

\subsection{Computational Complexity}

\begin{definition}
$TIME(f(n))$ ($NTIME(f(n))$) is the set of languages decidable within $O(f(n))$ steps on a single-tape deterministic (nondeterministic) TM. $\classnotation{P} = \bigcup_{k \ge 0}TIME(n^k)$, $\classnotation{NP} = \bigcup_{k \ge 0}NTIME(n^k)$. Decision on a NTM has that every computation branch halts, time is the number of transitions on the longest computation path, and space is the maximum number of cells visited on any computation path.
\end{definition}

\begin{theorem}
The following are members of $\classnotation{P}$:
\begin{itemize}
\item All regular languages
\item All context-free languages
\item $PATH = \{\langle G, s, t\rangle : G$ is an undirected graph having a path from $s$ to $t$\}
\end{itemize}
\end{theorem}

\begin{definition}
A \emph{verifier} is a TM that accepts a string $w$ and a \emph{certificate} $c$, and verifies whether $c$ is valid.
\end{definition}

\begin{theorem}
$\classnotation{NP}$ can also be defined as the set of languages with a polynomial-time verifier.
\end{theorem}

\begin{definition}
A language $A$ is \emph{polynomial-time reducible} to a language $B$--$A \le_p B$--if the reduction takes polynomial time.
\end{definition}

\begin{theorem}
Suppose $A \le_p B$. If $B \in \classnotation{P}$, then $A \in \classnotation{P}$; if $A \notin \classnotation{P}$, then $B \notin \classnotation{P}$.
\end{theorem}

\begin{definition}
A boolean formula $\phi$ in \emph{conjunctive normal form} is one that is a conjunction of clauses, and each clause is a disjunction of literals. A formula in \emph{3CNF} has $\le 3$ literals per clause. A formula is \emph{satisfiable} if there exists an assignment to the variables to make the formula true.
\end{definition}

\begin{theorem}
3SAT = $\{\langle \phi \rangle : \phi$ is a 3CNF formula that is satisfiable\} $\in \classnotation{NP}$, and 3SAT $\le_p$ CLIQUE = $\{\langle G, k\rangle : G$ is a graph with a $k$-clique\}.
\end{theorem}

\begin{definition}
A language $B$ is \emph{$\classnotation{NP}$-complete} if $B \in \classnotation{NP}$, and for every $A \in \classnotation{NP}$, $A \le_p B$. If only the second condition is true, then $B$ is \emph{$\classnotation{NP}$-hard}.
\end{definition}

\begin{theorem}
The following are $\classnotation{NP}$-complete:
\begin{itemize}
\item 3SAT (the Cook-Levin theorem)
\item CLIQUE
\item INDSET (same as CLIQUE but no edges between vertices)
\item VERTEX COVER (whether there exists a subset of vertices of size $\le k$ such that every edge involves a vertex in the subset)
\item HAMPATH (whether a directed graph contains a directed path through every vertex exactly once)
\item UHAMPATH (undirected version of HAMPATH)
\end{itemize}
\end{theorem}

\subsection{Space Complexity}

\begin{definition}
$\classnotation{PSPACE} = \bigcup_{k \ge 0} SPACE(n^k)$, $\classnotation{NPSPACE} = \bigcup_{k \ge 0} NSPACE(n^k)$.
\end{definition}

\begin{theorem}
For $f(n) \ge n$, $TIME(f(n)) \subseteq SPACE(f(n))$.
\end{theorem}

\begin{theorem}[Savitch's Theorem]
For $f(n) \ge n$, $NSPACE(f(n)) \subseteq SPACE(f^2(n))$.
\end{theorem}

\begin{corollary}
$\classnotation{PSPACE} = \classnotation{NPSPACE}$.
\end{corollary}

\begin{definition}
A language $B$ Is \emph{$\classnotation{PSPACE}$-complete} if $B \in \classnotation{PSPACE}$, and for every $A \in \classnotation{PSPACE}$, $A \le_p B$. If only the second condition is true, then $B$ is $\classnotation{PSPACE}$-hard.
\end{definition}

\begin{theorem}
The following are $\classnotation{PSPACE}$-complete:
\begin{itemize}
\item TQBF = \{$\langle\psi\rangle : \psi$ is a true quantified boolean formula\} (i.e., of the form $\psi = Q_1x_1\cdots Q_nx_n\phi(x_1,\cdots,x_n)$ where the $Q_i \in \{\exists, \forall\}$).
\item FORMULA-GAME (a 2-player version of TQBF, where players take turns choosing values for the variables in order)
\item Generalized Geography (a directed graph where each vertex is a string, and each edge has the next string start with the same letter as the previous one)
\item $A_{LBA}$
\end{itemize}
\end{theorem}

\begin{definition}
$\classnotation{L} = SPACE(\log(n))$, $\classnotation{NL} = NSPACE(\log(n))$.
\end{definition}

\begin{definition}
A \emph{log-space transducer} is a deterministic TM with read-only input, write-only output, and a read-write work tape, and the space it uses is equal to the length of the non-blank portion of the work tape + log(size of input) + log(size of output). A language $A$ is \emph{log-space reducible} to a language $B$ if there is a log-space transducer that computes a function $f$ for which $w \in B$ if and only if $f(w) \in A$. A language $B$ Is \emph{$\classnotation{NL}$-complete} if $B \in \classnotation{NL}$, and for every $A \in \classnotation{NL}$, $A \le_L B$. If only the second condition is true, then $B$ is $\classnotation{NL}$-hard.
\end{definition}

\begin{theorem}
PATH = \{$\langle G, s, t\rangle : G$ is a directed graph with a directed $s-t$ path\} is $\classnotation{NL}$-complete, and co$\classnotation{NL}$-complete.
\end{theorem}

\begin{corollary}
$\classnotation{NL}$ = co$\classnotation{NL}$
\end{corollary}

\begin{definition}
A function $f$ is \emph{space-constructible} if there is a TM that computes the function mapping $1^n$ (in unary) to $f(n)$ (in binary) in $O(f(n))$ space.
\end{definition}

\begin{theorem}[Space Hierarchy Theorem]
If $f$ is a space-constructible function, then there exists a language that can be decided in $O(f(n))$ space, but not in $o(f(n))$ space.
\end{theorem}

\begin{corollary}
$\classnotation{PSPACE} \ne \classnotation{EXPSPACE}$
\end{corollary}

\begin{corollary}
$\classnotation{NL} \ne \classnotation{PSPACE}$
\end{corollary}

\begin{definition}
A function $f$, which is $\Omega(n\log(n))$, is \emph{time-constructible} if there is a TM that computes the function mapping $1^n$ (in unary) to $f(n)$ (in binary) in $O(\frac{f(n)}{\log(n)})$ time.
\end{definition}

\begin{theorem}[Time Hierarchy Theorem]
If $f$ is a time-constructible function, then there exists a language that can be decided in $O(f(n))$ time, but not in $o(\frac{f(n)}{\log(f(n)})$ time.
\end{theorem}

\begin{corollary}
$\classnotation{P} \ne \classnotation{EXPTIME}$
\end{corollary}

\begin{corollary}
For any $1 < c < d$, $TIME(n^c) \ne TIME(n^d)$.
\end{corollary}

\begin{definition}
A language $B$ Is \emph{$\classnotation{EXPSPACE}$-complete} if $B \in \classnotation{EXPSPACE}$, and for every $A \in \classnotation{EXPSPACE}$, $A \le_p B$. If only the second condition is true, then $B$ is $\classnotation{EXPSPACE}$-hard.
\end{definition}

\begin{theorem}
$EQ_{REX\uparrow}$ = \{regular expressions with exponentiation\} is $\classnotation{EXPSPACE}$-complete.
\end{theorem}

\subsection{Relativized Complexity}

\begin{definition}
$\classnotation{P}^A$ (resp. $\classnotation{NP}^A$) = $\{L : L$ is decided by an oracle TM with an oracle for $A$ in deterministic (nondeterministic) polynomial time\}.
\end{definition}

\begin{theorem}
There exist oracles $A, B$ such that $\classnotation{P}^A = \classnotation{NP}^A$, and $\classnotation{P}^B \ne \classnotation{NP}^B$.
\end{theorem}