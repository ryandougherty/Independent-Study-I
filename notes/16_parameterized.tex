\section{Parameterized Complexity}

We know of $\P$ and $\NP$, and that there are many problems in each (and complete ones in $\NP$). However, we don't have any granularity when it comes to separating the two classes. A relatively recent technique is parameterization, which we discuss here. The goal is to find parameters of an input that make a particular language (i.e., problem) hard. 

\begin{definition}
A \emph{parameterized problem} $P$ is a set of tuples $(w, k)$ where $w$ is the \emph{input} and $k$ is the \emph{parameter} (the alphabets to which $w$ and $k$ belong are disjoint). We now define the pair of a language and the parameter:
\[
(L, k) = \{(w, m) : w \in L, m = k(w)\}.
\]
\end{definition}

We want to know whether $(w, k) \in P$ is true or not. The notion of a ``parameter" is basically a function that takes instances of a problem, and returns a non-negative integer. One possible parameter for vertex cover is: does there exist a vertex cover of size $\le k$?

\begin{definition}
We define the following classes (``solvable" means on a deterministic TM):
\begin{center}
$\XP = \{(L, k) : (L, k)$, for a given $w \in L$, is solvable in $|w|^{f(k)}$ time\}.\\
$\FPT = \{(L, k) : (L, k)$, for a given $w \in L$, is solvable in $|w|^{O(1)} \times f(k)$\}.
\end{center}
\end{definition}

\begin{theorem}
$\FPT$ can be redefined as $|w|^{O(1)} + f(k)$.
\end{theorem}

\begin{proof}
Going from addition to multiplication is trivially true. For the other direction:
\begin{itemize}
\item If $|w| \le f(k)$, then for a constant $c$, $f(k) \times |w|^c \le f(k)^{c+1}$.
\item If $|w| \ge f(k)$, then by a similar argument, $f(k) \times |w|^c \le |w|^{c+1}$.
\end{itemize}
Therefore, $f(k) \times |w|^c \le \max\{f(k)^{c+1}, |w|^{c+1}\} \le f(k)^{c+1} + |w|^{c+1}$. We can define this to be $f'(k) + n^{c'}$.
\end{proof}

\begin{theorem}
$\P \subseteq \FPT \subseteq \XP$.
\end{theorem}

\begin{proof}
The $\P$ problem is the same as a $\FPT$ problem that ignores its parameter. The second inclusion is obvious via the definition.
\end{proof}

Note that there is no bound on the growth of $f$; the only thing we must guarantee is that the running time of $M$ is polynomial in the length of $w$, if we assume a fixed parameter. This is why we usually do not study $\XP$, because the polynomial changes as $k$ changes (also, most languages are in $\XP$ anyway). Therefore, we study $\FPT$ by looking at the following problem:

\begin{theorem}
Define k-Vertex Cover $(KVC, k)$= $\{(G, k) : G = (V, E)$ is a graph and contains a vertex cover of size $k$\}. We have that $(KVC, k) \in \XP$, and $(KVC, k) \in \FPT$.
\end{theorem}

\begin{proof}
\par The parameter here is $k \le |V|$. We have that it is trivially in $\XP$, because we can check all possible $|V|^k$ subsets of the vertices, and each requires linear time in the number of vertices. Therefore, we have an $O(|V|^{k+1})$ algorithm (which has $f(k) = k+1$ here).

\par For showing it is in $\FPT$, we work as follows:
\begin{enumerate}
\item Choose an edge, and include one of the end vertices of the edge in the vertex cover. 
\item Delete all edges incident to the chosen vertex.
\item Repeat $k$ times.
\item Each repetition, check if there is a vertex cover; if not, repeat again. 
\end{enumerate}

In this algorithm, we end up checking $2^k$ possible different choices of vertices. Each step takes $O(|V|)$ time, so we have an $O(|V| \cdot 2^k)$ algorithm (having $f(k) = 2^k$), which shows that it is in $\FPT$.
\end{proof}

We can see that $\FPT$ is the parameterized analog of the decision class $\P$. However, how does one relate the two classes? We now define the \emph{slice} of a parameterized problem:
\begin{definition}
The \emph{$\ell$th slice} (for $\ell \in \mathbb{N}$) of a parameterized problem $(L, k)$ is:
\begin{center}
$(L, k)_{\ell} = \{x \in L : k(x) = \ell\}$.
\end{center}
\end{definition}

\newcommand{\paraCOLORABILITY}{\lang{paraColorability}}
The entire idea of slices is to separate a problem into many separate equivalence classes. For instance, one can define a problem $\paraCOLORABILITY$ as follows: given a graph $G$ and parameter $k \in \mathbb{N}$, decide whether $G$ is $k$-colorable. Obviously, the 3rd slice of the problem is the classical NP-complete problem $\lang{3COLORABILITY}$. By this reasoning, $\paraCOLORABILITY \notin \FPT$ unless $\P = \NP$.

\begin{theorem}
If $(L, k) \in \FPT$, then $(L, k)_{\ell} \in \P$.
\end{theorem}

\begin{proof}
Since $(L, k) \in \FPT$, $k$ is computable in polynomial time. Therefore, given $x \in Q$, we can determine if $k(x) = \ell$ in polynomial time (and equality check can be done in logarithmic time). Therefore, $(L, k)_{\ell} \in \P$.
\end{proof}

\subsection{Parameterized Reductions}

As we have done with complexity classes, we have a notion of parameterized reductions:

\begin{definition}
A \emph{parameterized reduction} (also called a $\FPT$ reduction) is a computable function $f$ that takes instances $(A, k_1)$ and maps them to $(B, k_2)$, where $A, B$ are decision languages, such that the following hold:
\begin{itemize}
\item $f$ is polynomial-time computable in terms of $k_1$,
\item $x \in (A, k_1) \Leftrightarrow y \in (B, k_2)$,
\item $k_2(y) \le g(k_1(x))$, where $g: \mathbb{N} \rightarrow \mathbb{N}$ is a computable function.
\end{itemize}
\end{definition}

In particular, we have that if $B$ (in the definition) is in $\FPT$, then $A$ is as well. Therefore, as we have done before, we will define hardness of $\FPT$.

\begin{theorem}
There exists a parameterized reduction from INDSET to CLIQUE, via $(G, k) \rightarrow (\overline{G}, k)$ ($\overline{G}$ is the complement graph of $G$).
\end{theorem}

\begin{proof}
Obvious via the definition.
\end{proof}

\begin{theorem}
$\FPT$ is closed under $\FPT$-reductions.
\end{theorem}

\begin{proof}
Let $M$ be an $\FPT$ reduction between the parameterized problems $(L_1, k_1)$ and $(L_2, k_2)$, and is computable in time $h(k_1(x)) \times p(|x|)$, and $k_2(y) \le g(k_1(x))$ for $y \in (L_2, k_2)$. 

\par Let $T$ be a TM that decides $L_2$ in time $f'(k_2(x)) \times p'(|x|)$, and assume $f'$ is non-decreasing. Then we can decide $L_1$ by computing $y$ and then decide if $y \in L_2$ via $T$, which can be done in time $h(k_1(x)) \times p(|x|) + f'(k_2(y)) \times p'(|y|) \le h(k_1(x)) \times q(|x|) + f'(g(k_1(x))) \times p'(h(k_1(x)) \times q(|x|))$. This show
\end{proof}

\newcommand{\paraNP}{\lang{paraNP}}
\subsection{$\paraNP$}
The class $\FPT$ is the parameterized analog of the decision class $\P$. But what about $\NP$? The equivalent class is $\paraNP$:
\begin{definition}
A parameterized problem $(L, k)$ is in $\paraNP$ if there is a computable function $f: \mathbb{N} \rightarrow \mathbb{N}$, a polynomial $p$, and a nondeterministic TM that, when given $x$, decides if $x \in L$ within $f(k(x)) \times p(|x|)$ steps.
\end{definition}

Now we make a connection between $\ell$th slices and $\FPT$:
\begin{theorem}
The following are equivalent:
\begin{enumerate}
\item $(L, k)$ is $\paraNP$-complete under $\FPT$ reductions.
\item Let $\ell, m_1, \cdots, m_{\ell} \in \mathbb{N}$. Then $\bigcup_{i=1}^{\ell} (L, k)_{m_i}$ is $\NP$-complete.
\end{enumerate}
\end{theorem}
\begin{proof}
For notation: let $k_{one}(x) = 1$ for all $x$. We can see that $(L, k_{one}) \in \FPT$ if and only if $L \in \P$.

\par For showing 1 implies 2, suppose that $(L, k)$ is $\paraNP$-complete under $\FPT$ reductions. Let $L'$ be any $\NP$-complete problem; we can see that $(L', k_{one}) \in \paraNP$. Let $R$ be an $\FPT$ reduction from $(L', k_{one}) \rightarrow (L, k)$. Let $p$ be the polynomial such that the reduction is done in $p(|x|)$ time. Let $f, g$ be the functions as defined above for $\FPT$-reductions (i.e., $R(x)$ is computable in time $f(k(x))\times p(|x|)$).

\par For any $x'$ in the language of $(L', k_{one})$, $R(x')$ is computable in $f(1)\times p(|x'|)$ time, with $k(R(x')) \le g(1)$. Therefore, $R$ is a poly-time reduction from $L'$ to $\bigcup_{i=1}^{g(1)}(L, k)_{i}$. Since $L'$ is $\NP$-complete by assumption, we have that this union is $\NP$-complete.
\end{proof}



\subsection{$\W$ Hierarchy}
As we have done with the polynomial hierarchy, we had a series of inclusions of classes. But what about parameterized complexity? There are quite a few hierarchies in parameterized complexity, but we give an introduction to one - the $\W$ hierarchy, as proposed by \cite{Downey1995109}. As a preliminary, we give some definitions:

\begin{definition}
A boolean circuit is called \emph{mixed-type} if it contains circuits that themselves have the following types of gates:
\begin{itemize}
\item \emph{Small gates} - $\neg, \vee, \wedge$ gates with bounded fan-in,
\item \emph{Large gates} - $\vee, \wedge$ gates with unbounded fan-in.
\end{itemize}
The \emph{depth} of a circuit is the longest path (using any type of gate) from an input to an output. We call the \emph{weft} of the circuit to be the maximum number of large gates from an input to an output. Also, a family of circuits has \emph{bounded depth} if every circuit in the family has depth at most a constant (and similarly for \emph{bounded weft}). A circuit family is \emph{monotone} if each circuit in the family does not have $\neg$-gates. A circuit is a \emph{decision circuit} if it has a single output.
\end{definition}

\begin{definition}
For a family of decision circuits $F$, we create the parameterized problem:
\[
L_F = \{(C, k) : C \in F, C \text{outputs 1 (``accept") some input of length} k\}.
\]
\end{definition}

\begin{definition}[$\W$ Hierarchy]
A parameterized problem $L$ is in $\W[t]$ if $L$ reduces to $L_F$ for a circuit family $F_{h, t}$, containing circuits of depth $h$ and weft at most $t$.

\par An equivalent definition is that a parameterized problem $L$ is in $\W[t]$ if there is a constant $c$ (may possibly be a function of $k$) and there is a parameterized reduction from $L$ to the following problem (called ``Weighted Circuit SAT"): given a boolean circuit and integer $k$, with weft at most $t$ and depth at most $c$, accept if there is an assignment of weight $k$ (number of ``true" inputs) and the output is 1.

\par Yet another equivalent definition for $\W[1]$ is the following: let $M$ be a single-tape nondeterministic TM. Let the problem $\lang{k-Step-NTM}$ be those single-tape nondeterministic TMs that accept with fewer than $k$ ``steps." We say that $L \in \W[1]$ if there is an $\FPT$ reduction to $\lang{k-Step-NTM}$, and is $\W[1]$-hard if there is an $\FPT$ reduction the other way. $L$ is $\W[1]$-complete if it is in $\W[1]$ and is $\W[1]$-hard (similarly for $\W[t]$ for $t \ge 2$).
\end{definition}

Therefore, we have the following containments (none is known to be proper):
\[
\FPT \subseteq \W[1] \subseteq \W[2] \subseteq \cdots \subseteq \XP
\]

Let's give some examples of problems that are $\W[1]$-complete. 

\begin{example}
$\lang{INDSET}$ is $\W[1]$-complete. Given $\langle G, k\rangle$, we construct a TM $M$ and $k' \in O(k^2)$ such that $M$ halts within $k'$ steps if and only if $G$ has an independent set of size $k$. Set the alphabet of the TM to be $G$'s vertices. For the first $k$ steps, $M$ guesses $k$ vertices for the first $k$ cells. For $1 \le i \le k$, $M$ moves to the $i$th cell, stores the corresponding vertex, and scans the tape to check if the other $k-1$ vertices are not adjacent to the $i$th one. Since $M$ does $k$ checks and each can be done in $O(k)$ time, we have $k' \in O(n^2)$. This shows $\lang{INDSET} \in \W[1]$.

\par For the other direction, given $\langle M, k\rangle$, we construct a graph $G$ that has an independent set of size $k^2$ if and only if $M$ halts within $k$ steps. We construct $G$ to have $k^2$ cliques: $K_1, \cdots K_{k^2}$. Select one vertex from each clique, $K_{i, j}$ (clique $i$, vertex $j$), which is the behavior of $M$ at step $i$ in cell $j$ in the computation (i.e., the configuration). Add edges between cliques to handle any non-determinism. Therefore, $\lang{INDSET}$ is $\W[1]$-hard. 
%The process is a reduction from $\lang{k-Step-NTM}$. We create a graph $G = (V, E)$ which has $k^2$ cliques (i.e., $k \times k$ in a ``grid"). The cliques $(x, y)$ correspond to the state of the NTM in cell $x$ at time $y$. Therefore, the size of $(x, y)$ is $O(n^2)$. Create an edge between two vertices if and only if these vertices correspond to two distinct configurations of the TM, including time in the computation. This shows that $\lang{INDSET}$ is $\W[1]$-hard.
%
%\par For the other direction, we guess $k$ vertices. For each pair, check if there is no edge between them. These $k$ vertices are written on $k$ cells of the TM tape. We then store the graph in a ``lookup table"
\end{example}