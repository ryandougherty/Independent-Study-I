\section{Polynomial Hierarchy, Alternating TMs}

\subsection{Polynomial Hierarchy}

From \Cref{thm:relativizePandNP}, we have a notion of using {\P} and {\NP} with the power of oracle machines. However, we don't have a generalization of a ``hierarchy" of such oracle machines (the theorem only concerns the ``first level"). Therefore, in \cite{originalpolyhierarchypaper}, the notion of a ``polynomial hierarchy" was created. The hierarchy is defined (equivalently) as follows:
\begin{itemize}
\item $\Sigma_1^\P$ = $\Pi_1^\P = \NP$,
\item $\Sigma_i^\P = \NP^{\Sigma_{i-1}^\P}$, 
\item $\Pi_i^\P = \coNP^{\Pi_{i-1}^\P}$.
\end{itemize}

\begin{figure}[!htb]
\label{fig:polyhierarchy}
\caption{Polynomial Hierarchy, taken from http://commons.wikimedia.org/wiki/File:Polynomial\_time\_hierarchy.svg. Each arrow represents inclusion: for example, $\Sigma_1^\P \subseteq \Sigma_2^\P$.}
\centering
\begin{tikzpicture}[->, node distance=2cm, semithick]
 \node (P) {$\Sigma_0^\P = \P = \Pi_0^\P$};
 \node (Sigma1) [above left of=P]       {$\NP = \Sigma_1^\P$ \hspace*{0.8cm}};
 \node (Pi1)    [above right of=P]      {\hspace*{1.2cm} $\Pi_1^\P$ = \coNP};
 \node (Sigma2) [above of=Sigma1]  {$\NP^\NP$ = $\Sigma_2^\P$ \hspace*{1.0cm}};
 \node (Pi2)    [above of=Pi1] {\hspace*{1.5cm} $\Pi_2^\P$ = $\coNP^\coNP$};
 \node (Sigma3) [above of=Sigma2]  {$\NP^{\NP^\NP}$ = $\Sigma_3^\P$ \hspace*{1.3cm}};
 \node (Pi3)    [above of=Pi2] {\hspace*{1.8cm} $\Pi_3^\P$ = $\coNP^{\coNP^\coNP}$};
 \node (dots)   [above right of=Sigma3]       {\vdots};
 \draw (P)      -> (Sigma1);
 \draw (P)      -> (Pi1);
 \draw (Sigma1) -> (Sigma2);
 \draw (Sigma1) -> (Pi2);
 \draw (Pi1)    -> (Pi2);
 \draw (Pi1)    -> (Sigma2);
 \draw (Sigma2) -> (Sigma3);
 \draw (Sigma2) -> (Pi3);
 \draw (Pi2)    -> (Pi3);
 \draw (Pi2)    -> (Sigma3);
\end{tikzpicture}
\end{figure}

%\begin{figure}
%\label{fig:polyhierarchy}
%\caption{Polynomial Hierarchy, taken from http://commons.wikimedia.org/wiki/File:Polynomial\_time\_hierarchy.svg. Each arrow represents inclusion: for example, $\Sigma_1^\P \subseteq \Delta_2^\P \subseteq \Sigma_2^\P$.}
%\centering
%\begin{tikzpicture}[->, node distance=2cm, semithick]
% \node (P) {$\Delta_0^\P = \Sigma_0^\P = \P = \Pi_0^\P = \Delta_1^\P$};
% \node (Sigma1) [above left of=P]       {$\NP = \Sigma_1^\P$ \hspace*{0.8cm}};
% \node (Pi1)    [above right of=P]      {\hspace*{1.2cm} $\Pi_1^\P$ = \coNP};
% \node (Delta2) [above left of=Pi1]     {$\P^\NP = \Delta_2^\P$ };
% \node (Sigma2) [above left of=Delta2]  {$\NP^\NP$ = $\Sigma_2^\P$ \hspace*{1.0cm}};
% \node (Pi2)    [above right of=Delta2] {\hspace*{1.5cm} $\Pi_2^\P$ = $\coNP^\NP$};
% \node (Delta3) [above left of=Pi2]     {$\P^{\NP^\NP} = \Delta_3^\P$};
% \node (Sigma3) [above left of=Delta3]  {$\NP^{\NP^\NP}$ = $\Sigma_3^\P$ \hspace*{1.3cm}};
% \node (Pi3)    [above right of=Delta3] {\hspace*{1.8cm} $\Pi_3^\P$ = $\coNP^{\NP^\NP}$};
% \node (dots)   [above of=Delta3]       {\vdots};
% \draw (P)      -> (Sigma1);
% \draw (P)      -> (Pi1);
% \draw (Sigma1) -> (Sigma2);
% \draw (Sigma1) -> (Delta2);
% \draw (Pi1)    -> (Pi2);
% \draw (Pi1)    -> (Delta2);
% \draw (Delta2) -> (Sigma2);
% \draw (Delta2) -> (Pi2);
% \draw (Sigma2) -> (Sigma3);
% \draw (Sigma2) -> (Delta3);
% \draw (Pi2)    -> (Pi3);
% \draw (Pi2)    -> (Delta3);
% \draw (Delta3) -> (Sigma3);
% \draw (Delta3) -> (Pi3);
%\end{tikzpicture}
%\end{figure}

\begin{definition}
The \emph{polynomial hierarchy}, called \PH, is defined to be:
\[
\PH = \bigcup_{k = 1}^{\infty} \Sigma_k^\P.
\]
\end{definition}

\begin{theorem}
$\PH$ can also be defined as:
\[
\PH = \bigcup_{k = 1}^{\infty} \Pi_k^\P.
\]
\end{theorem}

\begin{proof}
We have the simple inclusions: $\Sigma_k^\P \subseteq \Pi_{k+1}^\P \subseteq \Sigma_{k+2}^\P$.
\end{proof}

For $\PH$, we often call the $i$-th level of $\PH$ to be both $\Sigma_i^\P$ and $\Pi_i^\P$. A natural inclination, as has been done before with \NP, \PSPACE, \NL, and the like, is to find a complete problem for $\PH$. However, the following theorem shows that this is most likely not the case.

\begin{theorem}
If there is a $\PH$-complete language, then $\PH$ only has a finite number of levels.
\end{theorem}

\begin{proof}
Assume there exists a $\PH$-complete language $L$. By definition, $\PH = \bigcup_{k = 1}^{\infty} \Sigma_k^\P$. Therefore, for some $j$, we have that $L \in \Sigma_j^\P$, and every language in $\PH$ is polynomial-time reducible to $L$. However, any language $A$ that is polynomial-time reducible to some language $B \in \Sigma_j^\P$ has that $A \in \Sigma_j^\P$. Therefore, $\PH \subseteq \Sigma_j^\P$.
\end{proof}

This is quite a negative result compared to the complexity classes that we have studied earlier. However, for each $i$, there are many complete languages for $\Sigma_i^\P$ and $\Pi_i^\P$ (see \cite{Schaefer_completenessin}), by looking at a different definition of the hierarchy:
\begin{definition}
We define the following 2 languages:
\begin{itemize}
\item $\Sigma_i^{\SAT} = \{\langle \psi \rangle : \psi$ is of the form $\exists x_1 \forall x_2 \cdots Q_ix_i [\phi]$ where $\phi$ contains the variables $x_1,\cdots,x_i$, and $\psi$ is true\},
\item $\Pi_i^{\SAT} = \{\langle \psi \rangle : \psi$ is of the form $\forall x_1 \exists x_2 \cdots Q_ix_i [\phi]$ where $\phi$ contains the variables $x_1,\cdots,x_i$, and $\psi$ is true\}. 
\end{itemize}
\end{definition}

\begin{theorem}
$\Sigma_i^{\SAT}$ is $\Sigma_i^\P$-complete, and $\Pi_i^{\SAT}$ is $\Pi_i^\P$-complete.
\end{theorem}

\begin{proof}
Proven in \cite{Wrathall197623}.
\end{proof}

However, there is an alternate definition of the polynomial hierarchy that we will now use:
\begin{definition}
A language $L \in \Sigma_i^\P$ if there is a poly-time TM $M$ and a polynomial $p$ such that $x \in L$ if and only if:
\begin{center}
$\exists u_1 \in \{0, 1\}^{p(|x|)} \forall u_2 \in \{0, 1\}^{p(|x|)} \cdots Q_iu_i \in \{0, 1\}^{p(|x|)} M(x, u_1, \cdots, u_i)$ accepts.
\end{center}
where $Q_i$ is either $\exists$ or $\forall$ depending on whether $i$ is odd or not. Also, define $\PH$ as before.
\end{definition}

We prove that the definitions are equivalent. It is sufficient to show the following:

\begin{theorem}
For all $i \ge 2$, $\Sigma_i^\P = \NP^{\Sigma_{i-1}\SAT}$.
\end{theorem}

\begin{proof}
We can prove the $i=2$ case as higher $i$ values work the same way. So, we need to prove $\Sigma_2^\P$ (defined the new way) is equal to $\NP^{\SAT}$. If $L \in \Sigma_2^\P$, then there exists a poly-time TM $M$ and a polynomial $p$ such that $x \in L$ if and only if (by definition):
\begin{center}
$\exists u_1 \in \{0, 1\}^{p(|x|)} \forall u_2 \in \{0, 1\}^{p(|x|)} M(x, u_1, u_2)$ accepts.
\end{center}
The part ``$\forall u_2 \in \{0, 1\}^{p(|x|)} M(x, u_1, u_2)$ accepts" is exactly the negation of $\neg (\forall u_2 \in \{0, 1\}^{p(|x|)} M(x, u_1, u_2)$ accepts$)$, which is in $\NP$. We can use the $\SAT$ oracle for this. Therefore, there exists a NDTM $N$ that, given oracle access to $\SAT$, decides $L$ (by nondeterministically guessing $u_1$). This shows $\Sigma_2^\P \subseteq \NP^{\SAT}$.

\par Now we show $\NP^{\SAT} \subseteq \Sigma_2^\P$. Let $L \in \NP^{\SAT}$ - there exists a poly-time NDTM $N$ that decides $L$. Therefore, $x \in L$ if and only if there is a sequence of nondeterministic choices (and oracle answers) that has $N$ accept $x$. Let the choices be $c_1, \cdots, c_m \in \{0, 1\}$, and the answers be $a_1, \cdots, a_k \in \{0, 1\}$.

\par So, if $N$ uses these choices and receives $a_i$ from the oracle as the answer to the $i$th query made, then:
\begin{enumerate}
\item $M$ goes to $q_{accept}$
\item All answers are correct.
\end{enumerate}
Let $\phi_i$ be the $i$th query. Then, the second condition is equivalent to: if $a_i = 1$, then there is an assignment $u_i$ such that $\phi_i(u_i)$ is true, and for $a_i = 0$, then for all assignments $v_i$, $\phi_i(v_i)$ is false. Therefore, $x \in L$ if and only if: $\exists c_1, \cdots, c_m, a_1, \cdots, a_k, u_1, \cdots u_k \forall v_1, \cdots v_k$ such that:
\begin{enumerate}
\item $N$ accepts $x$ using $c_1, \cdots, c_m$ and $a_1, \cdots, a_k$, and:
\item If $a_i = 1$ then $\phi_i(u_i)$ is true for all $i$, and:
\item If $a_i = 0$ then $\phi_i(v_i)$ is false for all $i$.
\end{enumerate}
We can wrap the last two statements in another TM, so this implies $L \in \Sigma_2^\P$.
\end{proof}


\subsection{Alternating TMs}
\begin{definition}
\emph{Alternating TM}s (ATMs) are generalizations of NTMs, in that they behave the same as NTMs, but have an extra ``feature," that for every non-halting state, the state has a label from $\{\exists, \forall\}$. When running on some input, if the state has the $\exists$ label, then the ATM accepts if \emph{some} transition path from that state accepts; if the state has the $\forall$ label, then the ATM accepts if \emph{all} transition paths from that state accept.
\end{definition}

We need to make a distinction between ATMs that can alternate a limited and an unlimited number of times:

\begin{definition}
For all $i \in \mathbb{N}$, we say $L \in \Sigma_i\TIME(T(n))$ if there exists a $T(n)$-time ATM $M$ with initial state $\exists$ that recognizes $L$. Also, on every input possible and path from the starting configuration, $M$ alternates at most $i-1$ times (i.e., the state identifier changes, such as $\exists \rightarrow \forall$ or $\forall \rightarrow \exists$). We say $L \in \Pi_i\TIME(T(n))$ with the same definition as above, but the initial state is $\forall$ instead of $\exists$.
\end{definition}

Now we turn our attention to ATMs that can alternate unlimited number of times:

\newcommand{\ATIME}{\lang{ATIME}}
\newcommand{\TQBF}{\lang{TQBF}}
\begin{definition}
We say $L \in \ATIME(T(n))$ if there is an $O(T(n))$-time ATM $M$ that accepts $x$ if and only if $x \in L$. We set $\AP = \bigcup_{c \ge 0}\ATIME(n^c)$.
\end{definition}

We need to refine our definition of ``accepting" an input is, because of the quantifiers in the vertices. We define $G_{M, x}$ to be the DAG (directed acyclic graph), called the ``configuration graph," of $M$'s computation on $x$, where for any configuration $C$, there is an edge in $G_{M, x}$ from $C$ to configuration $C'$ if and only if $C'$ can be obtained from $C$ by 1 step of $M$'s execution. Therefore, accepting an input is defined as follows:
\begin{enumerate}
\item Configuration $C_{accept}$, if $M$ is in state $q_{accept}$, is labelled with a special name ``Accept."
\item If the machine is in an $\exists$ state, and there is an edge from $C$ to $C'$, and $C'$ is labelled ``Accept," then we label $C$ ``Accept."
\item Similarly for $C$ and any reachable configuration $C'$ from $C$, we label $C$ ``Accept."
\item We say $M$ accepts $x$ if, at the end of recursively applying rules, $C_{start}$ is labelled ``Accept."
\end{enumerate}

So how powerful is $\AP$? It seems that it can be very powerful! However, we show that it can only have polynomial space:
\begin{theorem}
$\AP = \PSPACE$.
\end{theorem}

\begin{proof}
For the $\TQBF$ problem, we can guess values for each $\exists$ variable using an $\exists$ state, and for each $\forall$ variable using a $\forall$ state. The poly-time computation is done at the very end. Therefore, $\TQBF \in \AP$, and since $\TQBF$ is $\PSPACE$-complete, we have $\PSPACE \subseteq \AP$.

\par For the other direction, we do a similar proof to $\NP \subseteq \PSPACE$. Let $L \in \AP$. Then, a ``game" (by making a game tree) is to see whether $x \in L$ has poly depth in the game tree. If the current node is a leaf (i.e., halting configuration), output the current value. Otherwise, recursively observe sub-game trees corresponding to the current node. 

\par We define two players $P_2$ and $P_2$. If the state is $\exists$, output that $P_1$ is the winner if and only if $P_1$ wins at least one of the sub-game trees. Otherwise, if the state is $\forall$, output that $P_2$ is the winner if and only if $P_2$ wins at least one of the sub-game trees. 

\par The algorithm is correct - it also runs in poly space because it only has poly depth, and each recursion we only need to store the current node's name, which is poly in length. Therefore, $L \in \PSPACE$.
\end{proof}

\subsection{Exercises}
\begin{enumerate}

\item Show that if $\NP = \coNP$, then $\PH = \NP$. Generalize this and show that if $\Sigma_k^\P = \Pi_k^\P$, then $\PH = \Sigma_k^\P$. % http://www.cs.umd.edu/~jkatz/complexity/f05/lecture6.pdf

\item We defined the polynomial hierarchy in terms of $\P$ - what about $\PSPACE$? Look at $\Sigma_k^\PSPACE, \Pi_k^\PSPACE$ and determine that $\PSPACE = \bigcup_{k \ge 1} \Sigma_k^\PSPACE = \bigcup_{k \ge 1} \Pi_k^\PSPACE$.
\end{enumerate}