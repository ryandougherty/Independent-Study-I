\section{Interactive Proofs}

We have talked before about $\NP$ and how we needed to provide a ``certificate" to prove that it truly is in the language (or is not, given a bad certificate). But what if we have a verifier and a prover that interact with each other? We can actually have more power for some canonical problems. This is called an ``interactive proof."

\begin{example}
Let's observe an interactive proof for $\phi \in \ThreeSAT$. For each clause in $\phi$, one at a time, have the prover give the values of the literals in the clause to the verifier, and have the verifier keep the answers until the end. At any point, if the prover gives conflicting values for the literals, reject. Otherwise, have the verifier check with the given values if all clauses are satisfied.
\end{example}

Now the last example works over each clause, so there are $n$ ``rounds" of interaction between the prover and verifier. Now we make this explicit:

\newcommand{\out}[2]{\texttt{out}$_{#1}\langle {#1}, {#2} \rangle(x)$}
\begin{definition}
A \emph{k-round} interaction of two functions $f, g$, with $k$ a non-negative integer (not necessarily independent of $f$ and $g$) is a sequence of strings $\{a_i\}_{i \ge 1}$:
\begin{itemize}
\item $a_1 = f(x)$
\item $a_2 = g(x, a_1)$
\item $\cdots$
\item $a_{2m+1} = f(x, a_1, \cdots, a_{2m})$, if $2m < k$
\item $a_{2m+2} = g(x, a_1, \cdots, a_{2m+1})$, if $2m+1 < k$.
\end{itemize}
We define the \emph{output} of $f$ (or $g$) to be $f(x, a_1, \cdots, a_k)$ at the end of the interaction (a common notation of this is \out{f}{g}). 
\end{definition}

Now we know what an interaction is, we need to know what a ``proof system" is:

\newcommand{\dIP}{\lang{dIP}}
\begin{definition}
$L$ is a \emph{k-round deterministic interactive proof system} if there is a $\poly(|x|)$-time TM $V$ that on input $x, a_1, \cdots, a_i$, has a $k$-round interaction with a function $f$ such that:
\begin{itemize}
\item If $x \in L$, then there is a function $f$ such that \out{V}{f} accepts.
\item If $x \notin L$, then for all functions $f$ \out{V}{f} rejects.
\end{itemize} 
Define $\dIP$ to be the set of languages with a $\poly(n)$-round deterministic interactive proof system. 
\end{definition}

\begin{theorem}
$\dIP = \NP$.
\end{theorem}

\begin{proof}
$\NP \subseteq \dIP$ because every $\NP$ language has a 1-round DPS. Suppose $L \in \dIP$, and $V$ is the verifier. Then, a certificate that $x \in L$ is a set of strings $\{a_i\}_{1 \le i \le k}$ that has $V$ accept. For verification, the verifier needs to check if $V(x) = a_1, V(x, a_1, a_2) = a_3$, and so on and that $V(x, a_1, \cdots, a_k)$ accepts. However, if the set of strings exists, then we can use a prover function $P$ such that $P(x, a_1) = a_2, P(x, a_1, a_2, a_3) = a_4$, and so on. This shows that \out{V}{P} accepts, meaning that $x \in L$. 
\end{proof}

\subsection{\IP}
So far we have talked about deterministic verifiers - what about probabilistic ones? We will introduce a class called $\IP$ which is very similar to $\BPP$:

\begin{definition}
We have $L \in \IP[k]$ for an integer $k \ge 1$ if there is a probabilistic poly-time TM $T$ with a $k$-round interaction such that:
\begin{itemize}
\item If $x \in L$, there exists a function $f$ such that $\Pr[$\out{T}{f} accepts$] \ge \frac{2}{3}$.
\item If $x \notin L$, for all functions $f$, $\Pr[$\out{T}{f} accepts$] \le \frac{1}{3}$
\end{itemize}
Define $\IP = \bigcup_{c \ge 1}\IP[n^c]$.
\end{definition}
As we have done with $\BPP$, we can success amplify to change the $\frac{2}{3}$ parameter to be exponentially close to 1, as well as $\frac{1}{3}$ to be exponentially close to 0. 

\par Let's work on an example: graph non-isomorphism. 
\begin{example}
We are given 2 graphs $G_1, G_2$. The verifier randomly and uniformly picks either of the graphs (say it is $G_i$), and permutes that graph's vertices. The permuted graph is sent to the prover. The prover identifies which of $G_1, G_2$ was sent. The number that the prover chooses is sent back to the verifier (say $j$). The verifier then accepts if $i = j$, and rejects if not.

\par If the graphs are not isomorphic, then there is a prover that has $\Pr[\text{verifier accepts}] = 1$ - the reason is that if a prover was all-powerful, it can tell which one of the graphs is isomorphic to the one sent. If the graphs are isomorphic, then the prover can only guess because any permutation of one graph is the same as that of the other graph. Therefore, $\Pr[\text{verifier accepts}] \le \frac{1}{2}$.
\end{example}

\newcommand{\CountSAT}{\lang{\#CountSAT}}
\begin{definition}
$\CountSAT = \{\langle \phi, k \rangle \colon \phi\;\text{is a 3-CNF formula with exactly $k$ satisfying assignments}\}$.
\end{definition}
We won't prove this here, but $\CountSAT$ is complete for a very powerful class, $\lang{\#P}$. We will use a technique commonly used for boolean formulas called \emph{arithmetization}. This technique turns a formula into polynomials. Given a 3-CNF formula $\phi$ with $m$ clauses and $n$ variables, create ``field variables" $X_1, \cdots, X_n$. For any clause $x_i \vee x_j \vee x_k$, we replace it with $Y_iY_jY_k$, where $Y_m$ is $X_m$ if $x_m$ is not complemented, and is $(1-X_m)$ otherwise. As an example, $x_1 \vee \overline{x_2} \vee x_3$ turns into $X_1(1-X_2)X_3$. 

\par Let the arithmetized version of the $j$th clause be $p_j(X_1, \cdots, X_n)$. We have $p_j(X_1, \cdots, X_n) = 1$ if an assignment to the $X_i$ makes the clause satisfiable, and 0 otherwise. 

\par Let $P_{\phi}(X_1, \cdots, X_n)$ be the product of the $p_i$ polynomials. This means $P_{\phi}$ is 1 if all the clauses are satisfied, and 0 otherwise. It has degree at most $3m$, and therefore has representation of size $O(m)$. 

\begin{theorem}
$\CountSAT \in \IP$.
\end{theorem}

\begin{proof}
Given $\phi$ and $k$, we construct $P_{\phi}$ as before. We see that the number of satisfying assignments to $\phi$ is:
\[
\sum_{b_1 \in \{0, 1\}, \cdots, b_n \in \{0, 1\}} P_{\phi}(b_1, \cdots, b_n)
\]
The prover $P$ wants to claim that the sum is exactly $k$. First, $P$ sends $V$ (the verifier) a prime $p$ between $2^n+1$ and $2^{2n}$. $V$ checks that $p$ is prime (using any prime-checking algorithm). All computations are done in $\mathbb{F}_p$ (integers $\mod p$). Since the sum above is between 0 and $2^n$, it has the same result over this field. We now use a general protocol called ``Sumcheck" to verify this equation.

\par The computation is as follows: given a polynomial $g$ of degree $d$ over $x_1, \cdots, x_n$, $k \in \mathbb{N}$, and $p$ a prime, we want to show that the prover can provide an interactive proof for $k = \sum_{b_1 \in \{0, 1\}} \cdots \sum_{b_n \in \{0, 1\}} g(x_1, \cdots, x_n)$ with all computations done modulo $p$. It is clear that given $b_1, \cdots, b_n$, the verifier can evaluate $g(b_1, \cdots, b_n)$ in poly-time.

\par Define $h(x_i) = \sum_{b_{i+1} \in \{0, 1\}} \cdots \sum_{b_n \in \{0, 1\}} g(x_i, b_{i+1}, \cdots, b_n)$. If the equation of equaling $k$ is true, then $h(0) + h(1) = k$. Consider this protocol:
\begin{enumerate}
\item If $n=1$, verify that $h(0) + h(1) = k$. If it is, then accept.
\item Otherwise, if $n \ge 2$, have the verifier ask the prover to send $h(x_1)$.
\item The prover sends some polynomial $s(x_1)$.
\item If $s(0) + s(1) \ne k$, reject (the prover sent the wrong polynomial). 
\item Otherwise, pick a random $a \in GF(p)$. Recursively call the protocol to check that\\ $s(a) = \sum_{b_2 \in \{0, 1\}} \cdots \sum_{b_n \in \{0, 1\}} g(a, b_2, \cdots, b_n)$. 
\end{enumerate}

We claim that if the original sum is not $k$, then we reject with probability $\ge (1-\frac{d}{p})^n$. We have the theorem from this because if the sum is true, then the prover can make the verifier accept with probability 1, and with a choice of $p$, $(1-\frac{d}{p})^n \approx 1-\frac{dn}{p}$, approximately 1. 

\par Assume the sum is false. For $n=1$, the verifier evaluates $g(0), g(1)$, and rejects with probability 1 if the sum is not $k$. Now assume that this is true for polynomials of degree $d$ with $n-1$ variables. In the first round, the prover should return the polynomial $h$. If so, then since $h(0) + h(1) \ne k$, the verifier automatically rejects. So suppose the prover returns $s(x_1) \ne h(x_1)$. Since $s(x_1) - h(x_1)$ has $\le d$ roots, there are $\le d$ values such that $s(a) = h(a)$. When we pick a random $a$, the probability that $s(a) \ne h(a)$ is $\ge 1-\frac{d}{p}$. 

\par If $s(a) \ne h(a)$, then the prover has an incorrect claim that he/she needs to prove when we recurse. By the induction hypothesis, the prover fails to prove this claim with probability $\ge (1-\frac{d}{p})^{n-1}$. Therefore, the probability that the verifier rejects is $\ge (1-\frac{d}{p})^n$. 
\end{proof}

\subsection{$\AM, \MA$}
Now we go to a motivating example for cryptography - how do we share items over the public internet without an attacker knowing? We use a similar example in complexity theory, called Arthur-Merlin protocols:

\begin{definition}
Define $\AM[k] \subseteq \IP[k]$ to be the class where we restrict the verifier's messages to be random bits, and the verifier must use these random bits (i.e., not part of the message). We have $\AM = \AM[2]$, which has the verifier send the first message (random string), the prover responds, and the verifier decides by applying a poly-time function to the entire transcript. Have $\MA$ be the same as $\AM$, but the prover sends the first message, the verifier computes a decision based off random coins and the transcript given by the prover.
\end{definition}

\subsection{$\IP = \PSPACE$}
This is a nontrivial result that was not expected. We knew the $\IP \subseteq \PSPACE$ direction before, and many people did not expect the other direction. 
\begin{theorem}
$\IP \subseteq \PSPACE$
\end{theorem}

\begin{proof}
Let $L \in \IP$, and let $V$ be the verifier, $P$ is the prover. We want to compute $\max_{P}\Pr[V \text{accepts} \leftrightarrow P \text{accepts $w$}$. Let this quantity be $x$. If $x \ge \frac{2}{3}$, then $w \in L$; if $x \le \frac{1}{3}$, then $w \notin L$ (by definition). The other values of $x$ cannot occur because $V$ is a verifier (all-powerful). All we need to do is to show how $x$ is computable in $\PSPACE$.

\par Suppose $V$ runs in $p(|w|)$ time, and chooses $p(|w|)$ random numbers. Just recursively simulate $V$ by branching on each random number, and each possible response by $P$. The recursion is polynomial, therefore this can be done in $\PSPACE$. We also keep track of the count of accepting branches given by $P$, and the total number of branches. Their length is polynomial in $|w|$. Therefore, the ratio is exactly $z$, which also can be done in $\PSPACE$.
\end{proof}

Now we will work in the other direction by making a protocol for $TQBF$, the $\PSPACE$-complete problem:
\begin{theorem}
$\PSPACE \subseteq \IP$.
\end{theorem}
\begin{proof}
We use arithmetization as before on the quantified boolean formula instance $\psi$. Therefore:
\begin{center}
$\Pi_{b_1 \in \{0, 1\}} \sum_{b_2 \in \{0, 1\}} \Pi_{b_3 \in \{0, 1\}} \cdots \sum_{b_n \in \{0, 1\}} p_{\psi}(b_1, \cdots, b_n) \ne 0$.
\end{center}
if and only if $\psi \in TQBF$. We observe that for any of the $b_i$, $b_i^j = b_i$ for all $j \ge 1$. Therefore, we can convert any polynomial $p_{\psi}$ into a multilinear polynomial $q_{\psi}$ that agrees with $p_{\psi}$ and has the degree of every $b_i$ at most 1. We define the linearization operator (which is linear in $x_i$) as follows:
\begin{center}
$L_i(p)(x_1, \cdots, x_n) = x_i \times p(x_1 \cdots, x_{i-1}, 1, x_{i+1}\cdots, x_n) + (1-x_i)\times p(x_1, \cdots, x_{i-1}, 0, x_{i+1}, \cdots, x_n)$
\end{center}
Therefore, $L_1(L_2(\cdots(L_n(p))\cdots))$ is a multilinear polynomial that agrees with $p$ on all $x_i$. Define $\forall x_i, \exists x_i$ as follows:
\begin{enumerate}
\item $\forall x_i p(x_1, \cdots, x_n) = p(x_1, \cdots, x_{i-1}, 0, x_{i+1}\cdots, x_n) \times p(x_1, \cdots, x_{i-1}, 1, x_{i+1}, \cdots, x_n)$.
\item $\exists x_i p(x_1, \cdots, x_n) = p(x_1, \cdots, x_{i-1}, 0, x_{i+1}\cdots, x_n) + p(x_1, \cdots, x_{i-1}, 1, x_{i+1}, \cdots, x_n)$.
\end{enumerate}
We can phrase the original $\psi \in TQBF$ sentence as: if we apply $\forall x_1 \exists x_2 \cdots \exists x_n$ on $p_{\psi}(x_1, \cdots, x_n)$, then the result is $k \ne 0$. Of course, we can add linearization operators in between without changing the result. We use the following expression:
\begin{center}
$\forall x_1 L_1 \exists x_2 L_1 L_2 \forall x_3 L_1 L_2 L_3 \cdots \exists x_n L_1 \cdots L_n p_{\psi}(x_1, \cdots, x_n)$
\end{center}
It's clear that this expression is poly-sized. Suppose for a polynomial $g$ on $x_1, \cdots, x_k$ that the prover can convince the verifier that $g(a_1, \cdots, a_k) = c$ with probability 1 for any $a_1, \cdots, a_k, c$ when this expression is true, and probability $< \epsilon$ when it is false. Let $u(x_1, \cdots x_{\ell}) = M_g(x_1, \cdots, x_k)$, where $M \in \{\exists x_i, \forall x_i, L_{i}\}$. We can see that $\ell = k-1$ for the first two, and $k$ for the last. 

\par Let $d \le 3m$ be the upper bound on the degree of $u$, and let the verifier know of $d$. We can show how the prover can convince the verifier that $u(a_1, \cdots, a_\ell) = f$ with probability 1 for any values when the expression is true and at most $\epsilon + \frac{d}{p}$ when it is false. 

\par Assume wlog that $i=1$. We have the following cases:
\begin{enumerate}
\item $M = \exists x_1$. The prover provides a polynomial $s(x_1)$ which he/she claims is $g(x_1, a_2, \cdots, a_k)$. The verifier checks (like in the $\CountSAT$ proof before) that $s(0) + s(1) = f$. If not, reject. Otherwise, the verifier picks a random $a \in GF(p)$ and asks the prover, recursively, to prove $s(a) = g(a, a_2, \cdots, a_k)$. 
\item $M = \forall x_1$. This is the same as the last case except we change the verification to be $s(0) \times s(1)$. 
\item $M = L_1$. The prover wants to prove $u(a_1, \cdots, u_k) = f$. The prover provides a polynomial $s(x_1)$ of degree $\le d$ that he/she claims is $g(x_1, a_2, \cdots, a_k)$. The verifier checks if $a_1 \times s(0) + (1-a_1) s(1) = f$. If not, reject. Otherwise, the verifier picks a random $a \in GF(p)$ and asks the prover, recursively, to prove $s(a) = g(a, a_2, \cdots, a_k)$. 
\end{enumerate}
The proof of correctness follows the same way as $\CountSAT$.
\end{proof}

