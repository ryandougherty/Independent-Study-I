\section{Algebraic Computation Models}

In computational models such as deterministic TMs, we look at a computational step as an application of a transition function, and then repeating until some end condition (or runs forever). Now we look at ``algebraic" computational models, which generalize classical computational models like $\P, \NP$ to mathematical fields, say the reals or complex numbers. 

\begin{definition}
Let $\mathbb{F}$ be a field. We call an \emph{algebraic straight program of length $T$} with inputs $x_i \in \mathbb{F}, 1 \le i \le n$, constants $c_i \in \mathbb{F}$ to be a sequence of $T$ statements of the form $a_i = b_{i_1} OP\; b_{i_2}$, where $OP$ is one of the field operations and the $b_{i_j}$ are inputs, constants, or a previous $a_i$. 
\end{definition}

\begin{theorem}
Any straight-line program of length $T$ with $n$ input variables is a polynomial with degree at most $2^T$.
\end{theorem}

\begin{proof}
Each input is of degree 1, and every step either adds or multiplies two polynomials. Since the produce of 2 polynomials has degree at most their sum, we have the result.
\end{proof}