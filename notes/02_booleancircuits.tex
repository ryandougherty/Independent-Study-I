\section{Boolean Circuits}

We have looked a little at boolean formulas, which take input variables that have values ``true" or ``false" and, through some operations, give a result of either ``true" or ``false" as ``output." Here, we look at circuits, which are a generalization of formulas.

\begin{definition}
A \emph{boolean circuit} is a DAG (directed acyclic graph) with a number of ``source" vertices (those with no incoming edges), also called ``gates," and a ``sink" vertex (with no outgoing edges), also called the ``output." Vertices are labelled with $\wedge, \vee, \neg$ - those with $\wedge, \vee$ have fan-in 2 and fan-out 1, and those with $\neg$ have fan-in and fan-out 1. The \emph{size} of a circuit is the number of vertices it has. The output of a particular input $x \in \{0, 1\}^n$ is applying each of the vertex ``rules" recursively until the output is reached. 
\end{definition}

\begin{theorem}
Any circuit with its $\wedge, \vee$ vertices having bounded fan-in (i.e., $\ge 3$) can be converted to an equivalent circuit with these vertices having only fan-in 2. 
\end{theorem}

\begin{definition}
A \emph{$T(n)$-size circuit family} is a sequence of circuits (i.e., $\{C_n\}_{n \in \mathbb{N}}$), where, for a given function $T \colon \mathbb{N} \rightarrow \mathbb{N}$, has that all of the circuits have size $\le T(n)$ for all $n$. We say that $L \in \SIZE(T(n))$ if there exists a $T(n)$-size circuit family such that for all $x \in \{0, 1\}^n$, $x \in L$ if and only if $C_n(x) = 1$ (i.e., ``accepts" $x$). 
\end{definition}

We start with an example: $L_1 = \{1^n \colon n \in \mathbb{Z}\}$.
\begin{theorem}
\label{thm:unary_linear_circuit_family}
$L_1$ can be decided by a linear-sized circuit family. 
\end{theorem}

\begin{proof}
The circuit is simply a tree of $\wedge$ gates that computes that of all inputs. 
\end{proof}

\begin{definition}
We define $\Ppoly$ to be $\bigcup_{c} \SIZE(n^c)$; in other words, $\Ppoly$ is the set of languages with poly-size circuit families.
\end{definition}

\begin{theorem}
\label{thm:all_unary_ppoly}
All unary languages are in $\Ppoly$.
\end{theorem}

\begin{proof}
Implied by \Cref{thm:unary_linear_circuit_family}.
\end{proof}

\begin{theorem}
\label{thm:p_subset_ppoly}
$\P \subseteq \Ppoly$, and the inclusion is strict.
\end{theorem}

\begin{proof}
Let $M$ be an ``oblivious" TM, which is one that has its head movement independent of the input. We need to show that for any $T(n)$-time oblivious TM, we can construct an equivalent $O(T(n))$-size circuit family. 

\par Let $x$ be an input for $M$, and look at the set of $M$'s configurations: $C_1, \cdots, C_{T(n)}$. Encode each configuration by a constant-size binary string. We can compute $C_i$ based on the following:
\begin{itemize}
\item $x$ itself,
\item $C_{i-1}$,
\item $C_{i_1}, \cdots, C_{i_k}$, where $C_{i_j}$ is the last step that $M$'s $j$th head was in the same position as it is in the $i$th step.
\end{itemize}
Because we assumed $M$ is oblivious, then $i_1, \cdots, i_k$ does not depend on $x$ - they only depend on $i$. Therefore, since we only have a constant number of strings of constant size, we can compute $C_i$ with a constant-size circuit.

\par We compose these circuits together so that, on input $x$, we compute $C_{T(n)}$ on $M$'s last step in its execution on $x$. We also have a constant size circuit that outputs 1 if and only if $C_{T(n)}$ is an accepting configuration. Since we have $O(T(n))$ circuits of constant size, we have constructed an equivalent $O(T(n))$-size circuit to $M$.

\par For showing strict containment, by \Cref{thm:all_unary_ppoly}, all undecidable unary languages are in $\Ppoly$, which are by definition not in $\P$.
\end{proof}



\newcommand{\CKTSAT}{\lang{CKT-SAT}}
\subsection{\CKTSAT}
As we proved the Cook-Levin Theorem from before, now we prove the circuit analog of $\SAT$. $\CKTSAT$ is the set of all circuits that have a ``satisfying assignment" to the $n$ input variables. In other words, there exists a $w \in \{0, 1\}^n$ if and only if $C(w)=1$ if $C \in \CKTSAT$.

\begin{theorem}
$\CKTSAT$ is $\NP$-complete.
\end{theorem}

\begin{proof}
$\CKTSAT$ is clearly in $\NP$. For hardness, if $L \in \NP$, then there is a verifier $V$ such that, on input $x$, verifies if $x \in L$ in polynomial time. By \Cref{thm:p_subset_ppoly}, we can convert $V$ into an equivalent circuit $C$ in polynomial time. Therefore, $x \in L$ if and only if $C \in \CKTSAT$.
\end{proof}

\subsection{Advice!}
We defined $\Ppoly$ in one way - now we will see another. This will make the fact that all undecidable unary languages are in $\Ppoly$ to be more clear. 

\begin{definition}
The set of languages \emph{decidable by $T(n)$-time TMs with $A(n)$ bits of advice} is called $\DTIME(T(n))/A(n)$. By this, we mean that for every $L \in \DTIME(T(n))/A(n)$, there is a sequence of strings, $\{a_n\}$, each of length $A(n)$, and a $T(n)$-time verifier $M$, such that it accepts $\langle x, a_n \rangle$ if and only if $x \in L$.
\end{definition}

In other words, we are given a sequence of strings of length $A(n)$ that are certificates, in a sense. Now we will look at the alternative definition of $\Ppoly$:
\begin{theorem}
$\Ppoly$ can be re-defined to be $\bigcup_{c, d \ge 0} \DTIME(n^c)/n^d$.
\end{theorem}

\begin{proof}
By the first definition of $\Ppoly$, if $L \in \Ppoly$, then $L$ is computable by a poly-size circuit family $\{C_n\}$. We can just use the descriptions of each of the $C_n$ as an advice string, and the TM is poly-time that, on input $\langle x, C \rangle$, where $C$ is an $n$-input circuit, outputs $C(x)$. 

\par If $L$ is decided by a poly-time TM $M$ with access to the sequence of advice strings $\{a_n\}$, each of poly-length, then we can use the construction of \Cref{thm:p_subset_ppoly} to obtain an equivalent poly-size circuit $B_n$ for all $n$ (i.e., outputs the same as $M$ does). Let $C_n$ be the same as $B_n$ but with $a_n$ hard-coded as the second input (i.e., fix the inputs). Therefore, $\{C_n\}$ is the desired circuit-family.
\end{proof}

We can actually get a ``nondeterministic" parallel to \Cref{thm:p_subset_ppoly}:

\begin{theorem}
$\NP \subseteq \bigcup_{c, d \ge 0} \NTIME(n^c)/n^d$, and the inclusion is strict (note $\NTIME$ instead of $\DTIME$).
\end{theorem}

\begin{proof}
It's easy to see inclusion: set $d = 0$, and have an $\NP$ machine accept a one-bit advice string for every $n$, and have the machine ignore the advice. 

\par For strict inclusion, we know that all unary languages are in $\DTIME(n)/1 \subseteq \NTIME(n)/1$, because we only need one bit of advice for each $n$ in advising the machine to check if $1^n$ is in the language. However, the following undecidable language is also in there, and therefore is not in $\NP$:
\begin{center}
$UHALT = \{1^n \colon n$'s binary expansion encodes $\langle M, w \rangle$ where $M$ halts on $w$\}.
\end{center}
\end{proof}

We have proved both of these, but how does $\NP$ relate to $\Ppoly$ (instead of the $\NTIME$ version)? We actually don't know for sure, as the polynomial hierarchy would collapse:

\begin{theorem}[Karp-Lipton]
If $\NP \subseteq \Ppoly$, then $\PH = \Sigma_2^\P$.
\end{theorem}

\begin{proof}
It is sufficient to show that $\NP \subseteq \Ppoly$ implies $\Sigma_3^\P \subseteq \Sigma_2^\P$. So fix a language $L \in \Sigma_3^\P$. By definition, there exists a poly-time TM $M$ and a polynomial $p$ such that if $x \in L$, then:
\begin{center}
$\exists u_1 \in \{0, 1\}^{p(|x|)} \forall u_2 \in \{0, 1\}^{p(|x|)} \exists u_3 \in \{0, 1\}^{p(|x|)} M(x, u_1, u_2, u_3)$ accepts.
\end{center}
We re-arrange this as follows with a new language $L'$: $\langle x, u_1, u_2 \rangle \in L'$ if and only if:
\begin{center}
$\exists u_3 \in \{0, 1\}^{p(|x|)} M(x, u_1, u_2, u_3)$ accepts.
\end{center}
Therefore, $L' \in \NP$ and therefore $L' \le_P \ThreeSAT$. Therefore, $\langle x, u_1, u_2 \rangle \in L'$ if and only if there exists a $\ThreeSAT$ formula $\phi$ over $x, u_1, u_2$ (reduction done in poly time). Also a NDTM $M'$ decides $L'$.

\par Now just plug in $M'$: $x \in L$ if and only if:
\begin{center}
$\exists u_1 \in \{0, 1\}^{p(|x|)} \forall u_2 \in \{0, 1\}^{p(|x|)} M'(x, u_1, u_2)$ accepts.
\end{center}
This is equivalent to:
\begin{center}
$\exists u_1 \in \{0, 1\}^{p(|x|)} \forall u_2 \in \{0, 1\}^{p(|x|)} \phi(x, u_1, u_2) \in \ThreeSAT$.
\end{center}
Now here is the central part of the proof. We assume that $\NP \subseteq \Ppoly$. Therefore, there exists a poly-size circuit $C$ for $\ThreeSAT$. We can just use $C$ to decide the expression above. So how do we find the circuit in the first place?

\par We prove the following. There exists a poly-time TM $M_C$ that, on input 3CNF formula $\phi$ with $|\phi| = n$ and a circuit $C$ with $n$ inputs such that: 
\begin{enumerate}
\item If $C$ recognizes $\ThreeSAT$ for inputs of length $n$ and $\phi \in \ThreeSAT$, then $M_C$ accepts.
\item If $\phi \notin \ThreeSAT$, then $M_C$ rejects.
\end{enumerate}
The idea is the following: repeatedly use the circuit $C$ to extract a satisfying assignment for $\phi$. Let the variables of $\phi$ be $x_1, ..., x_k$. If $C$ on input $\phi$ outputs 0, then $M_C$ rejects. Otherwise, fix $x_1$ to be false, and check if $C$ outputs 1 on the new circuit. If so, permanently fix $x_1$ to be false, and also have an array of $k$ values, and store $x_1$ to be the first element. Repeat for the other $k-1$ variables. Now we check: if $\phi$ with the assignments given have $\phi$ be true, then accept; otherwise, reject.

\par So now we go back to the original proof: we modify the statement above to be:
\begin{center}
$\exists C$ of poly-size $\exists u_1 \in \{0, 1\}^{p(|x|)} \forall u_2 \in \{0, 1\}^{p(|x|)} \M_C(C, \langle x, u_1, u_2\rangle)$ accepts.
\end{center}
Therefore, $L \in \Sigma_2^\P$.
\end{proof}

\subsection{$\AC, \NC$}
We have dealt with circuits of polynomial size, but what about the other direction? We define a similar analogue of $\L, \NL$ in space complexity but for circuits.

\begin{definition}
For $d \in \mathbb{N}$, we say $L \in \NC^d$ if L is decided by a family of poly-size, $O(\log^d n)$-depth circuits $\{C_n\}$, and $L \in \AC^d$ the same way but the gates have unbounded fan-in. We also define $\NC = \bigcup_{d \ge 0}\NC^d$ and $\AC = \bigcup_{d \ge 0}\AC^d$.
\end{definition}
So how much power does unbounded fan-in give? It only gives enough power to add a log-factor:

\begin{theorem}
$\NC^d \subseteq \AC^d \subseteq \NC^{d+1}$.
\end{theorem}
\begin{proof}
Unbounded fan-in can be simulated by a $O(\log n)$-depth tree of OR/AND gates.
\end{proof}

We also show that it is unlikely that any finite level of the $\NC^d$ hierarchy collapses (just like the $\PH$ hierarchy):
\begin{theorem}
$\NC^d = \NC^{d+1} \Rightarrow \NC = \NC^d$.
\end{theorem}

\begin{proof}
We show that $\NC^d = \NC^{d+1}$ implies $\NC^{d+1} = \NC^{d+2}$, from which the result follows. For a circuit $C$ of depth $O(\log^{d+2} n)$, partition it into $O(\log n)$ layers of size $O(\log^{d+1} n)$ each. Each partition $j$'s output is the input of $j+1$, and each contains polynomially many bits. Each bit at partition $j+1$ is computed from $j$ by a $\NC^{d+1}$ circuit. Since $\NC^d = \NC^{d+1}$, replace it by a $\NC^d$ circuit (may not be the same size, but can only be polynomially larger). This replacement has a circuit of poly-size, depth $O(\log^{d+1} n)$, and fan-in 2. Therefore, $\NC^{d+2} = \NC^{d+1} = \NC^d$. 
\end{proof}
Therefore, it is most likely that each inclusion of $\NC^d \subseteq \AC^d \subseteq \NC^{d+1}$ is strict (the only one proved, however, is $d = 0$).  So where does $\NC$ live in relation to other classes? We can show the following:
\begin{theorem}
$\NC^1 \subseteq \L$.
\end{theorem}

\begin{proof}
Given $x \in L$ with $|x| = n$, we construct the $n$th circuit of the circuit family, and then just evaluate the circuit from the output gate. Since we only need to record the path to the current gate, and any results along the path, and the fact that the circuit has log-depth, we only need log space to do this.
\end{proof}
As an immediate consequence, we have $\NC^1 \ne \PSPACE$. 

\begin{example}
We will show how to add 2 $n$-bit numbers in $\AC^0$ (the result is an $n+1$-bit number). Let the two numbers be $x = x_1\cdots x_n, y = y_1\cdots y_n$. Let $q_i = x_i \wedge y_i, p_i = x_i \vee y_i$ for all $i$. Initialize $c_1 = 0$, and have $c_i = \bigvee_{j=1}^{i-1} (q_j \wedge \bigwedge_{k=j+1}^{i-1}p_k)$ for $i \ge 2$. Have $z_i = x_i \oplus y_i \oplus c_i$ for all $i$, and $z_{n+1} = c_{n+1}$. The output is $z_1\cdots z_{n+1}$. 

\par We can compute $q_i, p_i$ at the same time (we only need 1 level). $c_i$ can be computed in 2 levels, and $z_i$ also in 2 levels. Therefore, we only need 5 levels, which is constant. Therefore, addition is in $\AC^0$. 
\end{example}

\subsection{Parity \& Switching Lemma}
Our goal here is to separate $\AC^0$ and $\NC^1$.
\newcommand{\ParityF}{\oplus}
\begin{definition}
Let $\ParityF(x_1, \cdots, x_n)$ be the function that computes the sum of the $x_i$ $(\mod 2)$.
\end{definition}
The goal here (using the Switching Lemma) is to prove:
\begin{theorem}
\label{thm:parity_not_in_ac0}
$\ParityF \notin \AC^0$.
\end{theorem}
H$\mathring{a}$stad proved the following (called the HSL):
\begin{theorem}
Let $f$ be a $k$-DNF formula (conjunction of disjunctions where each disjunction $\le k$ variables). Let $\rho$ be a random restriction by assigning random values to $t$ randomly selected input bits. Then for all $s \ge 2$, $\Pr_{\rho}[\text{$f$ restricted to $\rho$ is not expressible in CNF}] \le (\frac{(n-t)k^{10}}{n})^{s/2}$.
\end{theorem}
\begin{lemma}
HSL implies \Cref{thm:parity_not_in_ac0}.
\end{lemma}
\begin{proof}
Let $C$ be an $\AC^0$ circuit with a simplification: every vertex has out-degree 1, all $\neg$ gates are right after the input gates, the $\wedge, \vee$ gates alternate, and the bottom level has $\wedge$ gates with in-degree also 1. It is clear that we can construct this equivalent circuit with a polynomial number of gates.

\par Let the upper bound on the number of gates be $n^c$. We restrict more variables at each step of the computation: if there are $n_i$ unrestricted variables at step $i$, then we restrict $n-\sqrt{n_i}$ of them for the next step. Therefore, $n_i = \sqrt[2^i]{n}$. Let $c_i = 10c \times 2^i$. With high probability, we prove that at step $i$, we have a $(d-i)$-depth circuit, with $c_i$ fan-in at the lowest level. 

\par Suppose that the last level has $\wedge$ gates and the penultimate one has $\vee$ gates; it is easy to see that each $\vee$ gate computes a $c_i$-DNF. By HSL, with probability
\[
1-\sqrt{\left(\frac{c_i^{10}}{\sqrt[2^i+1]{n}}\right)^{c_{i+1}}} \ge 1 - \frac{1}{10n^c}
\]
for large-enough $n$, the function after the $(i+1)$-th iteration is expressible as a $c_{i+1}$-CNF. Since, for any CNF formula, the top gate is $\wedge$, we merge the $\wedge$ gates, which reduces the depth by 1. By the same reasoning, we can use HSL to transform the bottom-level $\vee$ gates, which has the level above it computes a $c_i$-CNF, into a $c_{i+1}$-DNF. We apply HSL at most once for each of the $n^c$ gates. Since by the union bound:
\[
\Pr[\bigcup_{i=1}^n A_i] \le \sum_{i=1}^n \Pr[A_i]
\]
we can achieve, with probability $\frac{9}{10}$, a depth-2 circuit with fan-in $c_{d-2}$ at the lowest level if we apply HSL $d-2$ times. The result is a $k$-CNF or DNF formula; we can make this constant by fixing $\le k$ variables (i.e., ensure the first clause in the DNF is true). Since $\ParityF$ is not constant under any restriction of $\le n$ variables, we have \Cref{thm:parity_not_in_ac0}.
\end{proof}
Now we prove the Switching Lemma:
\begin{proof}
Let a \emph{min-term} $m$ of function $f$ be a partial assignment that makes $f$ output 1 no matter what the assignment to the other variables are (i.e., the other variables are ``useless"); similarly, a \emph{max-term} does the same for $f$ but outputs 0 instead. We can see that every term in a $k$-CNF gives a $k$-size min-term, and every clause gives a $k$-size min-term. We can assume w.l.o.g. that the max and min-terms are minimal in size (i.e., no smaller max/min-terms possible). Therefore, a function not expressible in $s$-CNF must have at least 1 max-term of length $s+1$. 

\par For restrictions on disjoint sets of variables $A, B$, let $AB$ be the union of the restrictions.Let $R_t$ be the set of all restrictions on $t$ variables for $2t \ge n$ ($|R_t| = 2^t{n \choose t}$). Let $B$ be the restrictions $\rho \in R_t$ such that $f\vert_\rho$ is not expressible in $s$-CNF. We want to show $B$ is ``small." We show this by giving a 1-1 mapping between $B$ and $S = R_{t+s} \times \{0, 1\}^\ell$ for $\ell \in O(s \log k)$. We have $|S| = {n \choose t+s}2^{t+s}2^{O(s \log k)}$. Therefore,
\[
\frac{|B|}{|R_t|} = \frac{{n \choose t+s}k^{O(s)}}{{n \choose t}}
\]

\par This is an upper bound on the probability of choosing a ``bad" restriction. We can see from the presentation of HSL that $s, k$ are ``constants" in a sense. Therefore, this function is bounded by $n^{-\Omega(s)}$. 

\par We just need the 1-1 mapping now. Suppose $\rho$ is a ``bad" restriction on $f$ that is $k$-DNF. No terms of $f$ becomes 1 under $\rho$ (it becomes the constant function otherwise). Same reasoning goes for 0. Therefore, some terms become 0, but not all of them. Since we have $f\vert_\rho$ is not expressible in $s$-CNF, there exists a max-term $m$ with $|m| \ge s$. We can see that $f\vert_{\rho m}$ is the constant function 0, and $f\vert_{\rho m'}$ is nonzero for all sub-restrictions $m'$ of $m$. We want to show that this 1-1 mapping will have $\rho \rightarrow \rho\sigma$ for $\sigma$ an assignment to variables in $m$. Therefore, $\rho\sigma$ restricts $\ge t+s$ variables, as we want.

\par Give an order to $f$'s terms, and within those, order them in any order. We have $\rho m$ sets $f$ to be the 0-function. Split $m$ into $p \le s$ sub-restrictions: $m_1, \cdots m_p$, where each is defined as:
\begin{itemize}
\item Assume that we have an $i$ such that $m_1\cdots m_{i-1} \ne m$.
\item Let $t_{\ell_i}$ be the first term in our ordering that is not set to 0 under $\rho m_1\cdots m_{i-1}$ ($t_{\ell_i}$ exists since $m$ is a max-term, and $m_1\cdots m_{i-1}$ is not, since it is a proper subset).
\item Let $Y_i$ be the variables of $t_{\ell_i}$ set by $m$ but not by $\rho m_1\cdots m_{i-1}$. Since $m$ sets $t_{\ell_i}$ to 0, $Y_i$ is not empty. 
\item Let $a_i$ be the restriction of variables that agrees with $m$ on $Y_i$. Also, define $\sigma_i$ to be any restriction of $Y_i$ such that $t_{\ell_i}$ is not 0.
\end{itemize}
We continue until we assign at least $s$ variables, which happens when we get $m_1\cdots m_p$ (also, $\sigma_1 \cdots \sigma_p$). If we have assigned more than $s$ variables, change $m_p$ (i.e., fewer variables) so that we have exactly $s$ variables.

\par The mapping is $\rho \rightarrow (\rho\sigma_1\cdots\sigma_p, c)$, with $|c| \in O(s \log k)$ (defined later). Suppose we are given assignment $\rho\sigma_1\cdots\sigma_p$. We make this assignment to $f$, and observe which term is $t_{\ell_1}$ (the first term not fixed to 0 by $\rho$ - since we split $m$, this property holds for $\sigma_1$ and $\sigma_2, \cdots, \sigma_p$ do not assign values to terms in $t_{\ell_1}$). 

\par Let $s_1$ be the number of variables in $m_1$. $c$ contains $s_1$, indices in $t_{\ell_1}$ of the variables in $m_1$, and the values that $m_1$ assigns to those variables (we can see this is sufficient to know). 

\par Since each term has $\le k$ variables, we only need $O(s_1 \log k)$ bits to store this. After this is done, we change the restriction from $\rho\sigma_1\cdots\sigma_p$ to $\rho m_1 \sigma_2\cdots\sigma_p$, and then repeat the process for $s_2$. We, in this process, figure out what $m_1, \cdots, m_p$ are. 

\par We can now ``undo" them to reconstruct $\rho$, which shows the mapping is 1-1, and the total storage is $O(s \log k)$. 
\end{proof}

\subsection{$\ACC$ Circuits}
Sometimes we want even more power out of our circuits. This is precisely what the $\ACC$ hierarchy is for.
\begin{definition}
Let $MOD_m$ be the gate that has $m$ boolean inputs, and outputs 0 if the sum of the inputs is $0 (\mod m)$, and 1 otherwise. We have $L \in \ACC^0(m_1, \cdots, m_k)$ if there is a circuit family $\{C_n\}$ with $O(1)$ depth and poly-size, unbounded fan-in, and each has $\wedge, \vee, \neg$, and $MOD_{m_1}, \cdots, MOD_{m_k}$ gates that accepts $L$. We have $\ACC^0$ to be all of $\ACC^0(m_1, \cdots, m_k)$ for $k \ge 0$ and each $m_i > 0$. 
\end{definition}

\begin{theorem}
$\AC^0 \subseteq \ACC^0$, with strict inclusion.
\end{theorem}

\begin{proof}
Inclusion is obvious. We have strictness because $\ParityF \notin \AC^0$ above, and we have $\ParityF \in \ACC^0$ because we can just use 1 $MOD_2$ gate on all inputs.
\end{proof}

A recent result by Ryan Williams in 2010 shows that $\NEXP \not\subseteq \ACC^0$.

\subsection{$\TC$ Hierarchy}
Another hierarchy is the $\TC$ hierarchy, which instead of $MOD$ gates, has majority gates:
\begin{definition}
A \emph{majority gate} is one that on input $x_1, \cdots, x_n$, outputs 1 if the sum of the $x_i$ is $\ge \frac{n}{2}$, and 0 otherwise (i.e., a majority of the $x_i$ are 1). We have $\TC^i$ to be the class of poly-size, $O(\log^i(n))$-depth, unbounded fan-in circuits with majority gates. Also, we have $\TC$ to be the union of the $\TC^i$.
\end{definition}

\begin{theorem}
$\AC^0 \subseteq \TC^0$.
\end{theorem}
\begin{proof}
Let an $\wedge$ gate be called $AND$, $\vee$ for $OR$, and majority for $MAJ$. The value of $AND(x_1, \cdots, x_n)$ is exactly $MAJ(x_1, \cdots, x_n, 0, \cdots, 0)$, where we have $n$ zeroes. For $OR(x_1, \cdots, x_n)$ it is $MAJ(x_1, \cdots, x_n, 1, \cdots, 1)$, where there are $n$ ones. We can just hardwire the respective constants in, which does not change the depth or size. 
\end{proof}

We can actually define $\TC^0$ in terms of ``threshold" gates, which is called $T_k(x_1, \cdots, x_n)$, which outputs 1 if the sum of the $x_i$ is at least $k$, and 0 otherwise. Note that this is a generalization of majority gates. We can actually simulate any threshold gate with majority gates:
\begin{itemize}
\item $T_k(x_1, \cdots, x_n)$ is precisely $MAJ(x_1, \cdots, x_n, 0, \cdots, 0)$ if $k \ge \frac{n}{2}$, and there are $2k-n$ zeroes,
\item $T_k(x_1, \cdots, x_n)$ is precisely $MAJ(x_1, \cdots, x_n, 1, \cdots, 1)$ otherwise, and there are $n-2k$ ones.
\end{itemize}

\begin{theorem}
$\TC^0 \subseteq \NC^1$. 
\end{theorem}
\begin{proof}
Each of the threshold gates can be substituted for a $\NC^1$ circuit ($O(\log n)$-depth). Since $\TC^0$ circuits have constant depth, the resulting circuit has $O(\log n)$ depth and bounded fan-in.
\end{proof}

\begin{corollary}
$\NC = \AC = \TC$.
\end{corollary}

\subsection*{Exercises}

\begin{enumerate}
\item A language $L \subseteq \{0, 1\}^*$ is sparse if there is a polynomial $p$ such that $|L \cap \{0, 1\}^n| \le p(n)$ for every $n \in \mathbb{N}$. Show that every sparse language is in $\Ppoly$.
\item Going off the previous question, show that if a sparse language is $\NP$-complete, then $\P = \NP$.
\item Let $\L^i$ be the languages decided by $O(\log^i n)$-space TMs, and $\NL^i$ the same way for NTMs.
\begin{enumerate}
\item Show that $\NC^i \subseteq \L^i$ for all $i$. % final/final-soln.pdf
\item Show that if $\NL^i \subseteq \NC^{2i}$ for all $i$, then we solve a major open problem. Note that this is true for $i=1$. % final/final-soln.pdf
\end{enumerate}
\end{enumerate}