% This is LLNCS.DOC the documentation file of
% the LaTeX2e class from Springer-Verlag
% for Lecture Notes in Computer Science, version 2.4
\documentclass{article}
\usepackage{cleveref}
%\usepackage{llncsdoc}
\usepackage[margin=1in]{geometry}
\usepackage{amsfonts}
\usepackage{complexity}
\usepackage{tikz}
\usepackage{fancyhdr}
\usepackage{amsmath}
\usetikzlibrary{arrows}
\usepackage{amsthm}

\theoremstyle{definition}
\newtheorem{theorem}{Theorem}
\numberwithin{theorem}{subsection} % important bit

\newtheorem{definition}[theorem]{Definition}
\newtheorem{corollary}[theorem]{Corollary}
\newtheorem{lemma}[theorem]{Lemma}

\pagestyle{fancy}
\renewcommand{\sectionmark}[1]{\markright{#1}}
\fancyhf{}
\rhead{\fancyplain{}{Ryan Dougherty}}
\lhead{\fancyplain{}{\rightmark }} 
\cfoot{\fancyplain{}{\thepage}}

% Comments
\def\Comment#1{{\color{red} \textsf{\textsl{$\langle\!\langle$#1\/$\rangle\!\rangle$}}}}

\begin{document}
\markboth{Independent Study -- Complexity Theory}{Independent Study -- Complexity Theory}
\thispagestyle{empty}

\rule{\textwidth}{1pt}
\vspace{2pt}
\begin{flushright}
\Huge
\begin{tabular}{@{}l}
Independent Study \\ Complexity Theory\\[6pt]
{\large Ryan Dougherty}
\end{tabular}
\end{flushright}
\rule{\textwidth}{1pt}

\tableofcontents

\newcommand{\dir}{notes}

\section{Introduction \& Preface}

Welcome to this series of lecture notes! The main book that the material comes from is Arora and Barak's \emph{Computational Complexity} book \cite{arora_baraz_computational_complexity}. Some material that is assumed from the reader (and is referenced in \Cref{sec:review}) is from Sipser's \emph{Introduction to the Theory of Computation} book \cite{sipsertheoryofcomp}. We assume that the reader has a reasonable understanding of the following material:
\begin{itemize}
\item \{Regular, Context-free, Turing-decidable, Turing-recognizable\} languages, and their machine counterparts
\item (Un)decidability
\item Reducibility
\item Recursion theorem
\item Time complexity: $\p, \np, \exptime$, and their -complete versions
\item Space complexity: $\pspace, \npspace, \expspace, \logspace, \nlogspace$, and their -complete versions
\end{itemize}
\section{Review}
\label{sec:review}

This section highlights many of the key definitions and theorems studied in a first-year graduate (or advanced undergraduate) course in complexity theory. We assume the reader knows about finite automata (DFAs/NFAs), grammars (CFGs), and Turing machines (TMs), and their respective language classes.

\begin{definition}
A TM is a \emph{decider} if it halts (accepts or rejects) on every input.
\end{definition}

\begin{definition}
A language $B$ is \emph{decidable} if there exists a decider $D$ such that $L(D) = B$. A language $C$ is \emph{undecidable} if $C$ is not decidable.
\end{definition}

\begin{theorem}
The following are decidable:
\begin{itemize}
\item $A_{DFA} = \{\langle M, w \rangle : M$ is a DFA that accepts $w$\}.
\item $E_{DFA} = \{\langle M \rangle : M$ is a DFA whose language is empty\}.
\item $ALL_{DFA} = \{\langle M \rangle : M$ is a DFA whose language is $\Sigma^*$\}.
\item $EQ_{DFA} = \{\langle M_1, M_2 \rangle : M_1$ and $M_2$ are DFAs and $L(M_1) = L(M_2)$\}.
\item $A_{CFG} = \{\langle G, w \rangle : G$ is a CFG that generates $w$\}.
\item $E_{CFG} = \{\langle G \rangle : L(G)$ is empty\}.
\end{itemize}
\end{theorem}

\begin{theorem}
The following are undecidable:
\begin{itemize}
\item $ALL_{CFG} = \{\langle G \rangle : G$ is a CFG and $L(G) = \Sigma^*$\}.
\item $EQ_{CFG} = \{\langle G_1, G_2 \rangle : G_1$ and $G_2$ are CFGs and $L(G_1) = L(G_2)$\}.
\item $A_{TM} = \{\langle M, w \rangle : M$ is a TM that accepts $w$\}.
\end{itemize}
\end{theorem}
\section{Polynomial Hierarchy, Alternating TMs}

\subsection{Polynomial Hierarchy}

From \Cref{thm:relativizePandNP}, we have a notion of using {\P} and {\NP} with the power of oracle machines. However, we don't have a generalization of a ``hierarchy" of such oracle machines (the theorem only concerns the ``first level"). Therefore, in \cite{originalpolyhierarchypaper}, the notion of a ``polynomial hierarchy" was created. The hierarchy is defined (equivalently) as follows:
\begin{itemize}
\item $\Sigma_0^\P$ = $\Pi_0^\P = \P$,
\item $\Sigma_i^\P = \NP^{\Sigma_{i-1}^\P}$, 
\item $\Pi_i^\P = \coNP^{\Pi_{i-1}^\P}$.
\end{itemize}

\begin{figure}[!htb]
\label{fig:polyhierarchy}
\caption{Polynomial Hierarchy, taken from http://commons.wikimedia.org/wiki/File:Polynomial\_time\_hierarchy.svg. Each arrow represents inclusion: for example, $\Sigma_1^\P \subseteq \Sigma_2^\P$.}
\centering
\begin{tikzpicture}[->, node distance=2cm, semithick]
 \node (P) {$\Sigma_0^\P = \P = \Pi_0^\P$};
 \node (Sigma1) [above left of=P]       {$\NP = \Sigma_1^\P$ \hspace*{0.8cm}};
 \node (Pi1)    [above right of=P]      {\hspace*{1.2cm} $\Pi_1^\P$ = \coNP};
 \node (Sigma2) [above of=Sigma1]  {$\NP^\NP$ = $\Sigma_2^\P$ \hspace*{1.0cm}};
 \node (Pi2)    [above of=Pi1] {\hspace*{1.5cm} $\Pi_2^\P$ = $\coNP^\coNP$};
 \node (Sigma3) [above of=Sigma2]  {$\NP^{\NP^\NP}$ = $\Sigma_3^\P$ \hspace*{1.3cm}};
 \node (Pi3)    [above of=Pi2] {\hspace*{1.8cm} $\Pi_3^\P$ = $\coNP^{\coNP^\coNP}$};
 \node (dots)   [above right of=Sigma3]       {\vdots};
 \draw (P)      -> (Sigma1);
 \draw (P)      -> (Pi1);
 \draw (Sigma1) -> (Sigma2);
 \draw (Sigma1) -> (Pi2);
 \draw (Pi1)    -> (Pi2);
 \draw (Pi1)    -> (Sigma2);
 \draw (Sigma2) -> (Sigma3);
 \draw (Sigma2) -> (Pi3);
 \draw (Pi2)    -> (Pi3);
 \draw (Pi2)    -> (Sigma3);
\end{tikzpicture}
\end{figure}

%\begin{figure}
%\label{fig:polyhierarchy}
%\caption{Polynomial Hierarchy, taken from http://commons.wikimedia.org/wiki/File:Polynomial\_time\_hierarchy.svg. Each arrow represents inclusion: for example, $\Sigma_1^\P \subseteq \Delta_2^\P \subseteq \Sigma_2^\P$.}
%\centering
%\begin{tikzpicture}[->, node distance=2cm, semithick]
% \node (P) {$\Delta_0^\P = \Sigma_0^\P = \P = \Pi_0^\P = \Delta_1^\P$};
% \node (Sigma1) [above left of=P]       {$\NP = \Sigma_1^\P$ \hspace*{0.8cm}};
% \node (Pi1)    [above right of=P]      {\hspace*{1.2cm} $\Pi_1^\P$ = \coNP};
% \node (Delta2) [above left of=Pi1]     {$\P^\NP = \Delta_2^\P$ };
% \node (Sigma2) [above left of=Delta2]  {$\NP^\NP$ = $\Sigma_2^\P$ \hspace*{1.0cm}};
% \node (Pi2)    [above right of=Delta2] {\hspace*{1.5cm} $\Pi_2^\P$ = $\coNP^\NP$};
% \node (Delta3) [above left of=Pi2]     {$\P^{\NP^\NP} = \Delta_3^\P$};
% \node (Sigma3) [above left of=Delta3]  {$\NP^{\NP^\NP}$ = $\Sigma_3^\P$ \hspace*{1.3cm}};
% \node (Pi3)    [above right of=Delta3] {\hspace*{1.8cm} $\Pi_3^\P$ = $\coNP^{\NP^\NP}$};
% \node (dots)   [above of=Delta3]       {\vdots};
% \draw (P)      -> (Sigma1);
% \draw (P)      -> (Pi1);
% \draw (Sigma1) -> (Sigma2);
% \draw (Sigma1) -> (Delta2);
% \draw (Pi1)    -> (Pi2);
% \draw (Pi1)    -> (Delta2);
% \draw (Delta2) -> (Sigma2);
% \draw (Delta2) -> (Pi2);
% \draw (Sigma2) -> (Sigma3);
% \draw (Sigma2) -> (Delta3);
% \draw (Pi2)    -> (Pi3);
% \draw (Pi2)    -> (Delta3);
% \draw (Delta3) -> (Sigma3);
% \draw (Delta3) -> (Pi3);
%\end{tikzpicture}
%\end{figure}

\begin{definition}
The \emph{polynomial hierarchy}, called \PH, is defined to be:
\[
\PH = \bigcup_{k = 1}^{\infty} \Sigma_k^\P.
\]
\end{definition}

\begin{theorem}
$\PH$ can also be defined as:
\[
\PH = \bigcup_{k = 1}^{\infty} \Pi_k^\P.
\]
\end{theorem}

\begin{proof}
We have the simple inclusions: $\Sigma_k^\P \subseteq \Pi_{k+1}^\P \subseteq \Sigma_{k+2}^\P$.
\end{proof}

For $\PH$, we often call the $i$-th level of $\PH$ to be both $\Sigma_i^\P$ and $\Pi_i^\P$. A natural inclination, as has been done before with \NP, \PSPACE, \NL, and the like, is to find a complete problem for $\PH$. However, the following theorem shows that this is most likely not the case.

\begin{theorem}
If there is a $\PH$-complete language, then $\PH$ only has a finite number of levels.
\end{theorem}

\begin{proof}
Assume there exists a $\PH$-complete language $L$. By definition, $\PH = \bigcup_{k = 1}^{\infty} \Sigma_k^\P$. Therefore, for some $j$, we have that $L \in \Sigma_j^\P$, and every language in $\PH$ is polynomial-time reducible to $L$. However, any language $A$ that is polynomial-time reducible to some language $B \in \Sigma_j^\P$ has that $A \in \Sigma_j^\P$. Therefore, $\PH \subseteq \Sigma_j^\P$.
\end{proof}

This is quite a negative result compared to the complexity classes that we have studied earlier. However, for each $i$, there are many complete languages for $\Sigma_i^\P$ and $\Pi_i^\P$ (see \cite{Schaefer_completenessin}), by looking at a different definition of the hierarchy:
\begin{definition}
We define the following 2 languages:
\begin{itemize}
\item $\Sigma_i^{\SAT} = \{\langle \psi \rangle : \psi$ is of the form $\exists x_1 \forall x_2 \cdots Q_ix_i [\phi]$ where $\phi$ contains the variables $x_1,\cdots,x_i$, and $\psi$ is true\},
\item $\Pi_i^{\SAT} = \{\langle \psi \rangle : \psi$ is of the form $\forall x_1 \exists x_2 \cdots Q_ix_i [\phi]$ where $\phi$ contains the variables $x_1,\cdots,x_i$, and $\psi$ is true\}. 
\end{itemize}
\end{definition}

\begin{theorem}
$\Sigma_i^{\SAT}$ is $\Sigma_i^\P$-complete, and $\Pi_i^{\SAT}$ is $\Pi_i^\P$-complete.
\end{theorem}

\begin{proof}
Proven in \cite{Wrathall197623}.
\end{proof}

However, there is an alternate definition of the polynomial hierarchy that we will now use:
\begin{definition}
A language $L \in \Sigma_i^\P$ if there is a poly-time TM $M$ and a polynomial $p$ such that $x \in L$ if and only if:
\begin{center}
$\exists u_1 \in \{0, 1\}^{p(|x|)} \forall u_2 \in \{0, 1\}^{p(|x|)} \cdots Q_iu_i \in \{0, 1\}^{p(|x|)} M(x, u_1, \cdots, u_i)$ accepts.
\end{center}
where $Q_i$ is either $\exists$ or $\forall$ depending on whether $i$ is odd or not. Also, define $\PH$ as before.
\end{definition}

We prove that the definitions are equivalent. It is sufficient to show the following:

\begin{theorem}
For all $i \ge 2$, $\Sigma_i^\P = \NP^{\Sigma_{i-1}\SAT}$.
\end{theorem}

\begin{proof}
We can prove the $i=2$ case as higher $i$ values work the same way. So, we need to prove $\Sigma_2^\P$ (defined the new way) is equal to $\NP^{\SAT}$. If $L \in \Sigma_2^\P$, then there exists a poly-time TM $M$ and a polynomial $p$ such that $x \in L$ if and only if (by definition):
\begin{center}
$\exists u_1 \in \{0, 1\}^{p(|x|)} \forall u_2 \in \{0, 1\}^{p(|x|)} M(x, u_1, u_2)$ accepts.
\end{center}
The part ``$\forall u_2 \in \{0, 1\}^{p(|x|)} M(x, u_1, u_2)$ accepts" is exactly the negation of $\neg (\forall u_2 \in \{0, 1\}^{p(|x|)} M(x, u_1, u_2)$ accepts$)$, which is in $\NP$. We can use the $\SAT$ oracle for this. Therefore, there exists a NDTM $N$ that, given oracle access to $\SAT$, decides $L$ (by nondeterministically guessing $u_1$). This shows $\Sigma_2^\P \subseteq \NP^{\SAT}$.

\par Now we show $\NP^{\SAT} \subseteq \Sigma_2^\P$. Let $L \in \NP^{\SAT}$ - there exists a poly-time NDTM $N$ that decides $L$. Therefore, $x \in L$ if and only if there is a sequence of nondeterministic choices (and oracle answers) that has $N$ accept $x$. Let the choices be $c_1, \cdots, c_m \in \{0, 1\}$, and the answers be $a_1, \cdots, a_k \in \{0, 1\}$.

\par So, if $N$ uses these choices and receives $a_i$ from the oracle as the answer to the $i$th query made, then:
\begin{enumerate}
\item $M$ goes to $q_{accept}$
\item All answers are correct.
\end{enumerate}
Let $\phi_i$ be the $i$th query. Then, the second condition is equivalent to: if $a_i = 1$, then there is an assignment $u_i$ such that $\phi_i(u_i)$ is true, and for $a_i = 0$, then for all assignments $v_i$, $\phi_i(v_i)$ is false. Therefore, $x \in L$ if and only if: $\exists c_1, \cdots, c_m, a_1, \cdots, a_k, u_1, \cdots u_k \forall v_1, \cdots v_k$ such that:
\begin{enumerate}
\item $N$ accepts $x$ using $c_1, \cdots, c_m$ and $a_1, \cdots, a_k$, and:
\item If $a_i = 1$ then $\phi_i(u_i)$ is true for all $i$, and:
\item If $a_i = 0$ then $\phi_i(v_i)$ is false for all $i$.
\end{enumerate}
We can wrap the last two statements in another TM, so this implies $L \in \Sigma_2^\P$.
\end{proof}


\subsection{Alternating TMs}
\begin{definition}
\emph{Alternating TM}s (ATMs) are generalizations of NTMs, in that they behave the same as NTMs, but have an extra ``feature," that for every non-halting state, the state has a label from $\{\exists, \forall\}$. When running on some input, if the state has the $\exists$ label, then the ATM accepts if \emph{some} transition path from that state accepts; if the state has the $\forall$ label, then the ATM accepts if \emph{all} transition paths from that state accept.
\end{definition}

We need to make a distinction between ATMs that can alternate a limited and an unlimited number of times:

\begin{definition}
For all $i \in \mathbb{N}$, we say $L \in \Sigma_i\TIME(T(n))$ if there exists a $T(n)$-time ATM $M$ with initial state $\exists$ that recognizes $L$. Also, on every input possible and path from the starting configuration, $M$ alternates at most $i-1$ times (i.e., the state identifier changes, such as $\exists \rightarrow \forall$ or $\forall \rightarrow \exists$). We say $L \in \Pi_i\TIME(T(n))$ with the same definition as above, but the initial state is $\forall$ instead of $\exists$.
\end{definition}

Now we turn our attention to ATMs that can alternate unlimited number of times:

\newcommand{\ATIME}{\lang{ATIME}}
\newcommand{\TQBF}{\lang{TQBF}}
\begin{definition}
We say $L \in \ATIME(T(n))$ if there is an $O(T(n))$-time ATM $M$ that accepts $x$ if and only if $x \in L$. We set $\AP = \bigcup_{c \ge 0}\ATIME(n^c)$.
\end{definition}

We need to refine our definition of ``accepting" an input is, because of the quantifiers in the vertices. We define $G_{M, x}$ to be the DAG (directed acyclic graph), called the ``configuration graph," of $M$'s computation on $x$, where for any configuration $C$, there is an edge in $G_{M, x}$ from $C$ to configuration $C'$ if and only if $C'$ can be obtained from $C$ by 1 step of $M$'s execution. Therefore, accepting an input is defined as follows:
\begin{enumerate}
\item Configuration $C_{accept}$, if $M$ is in state $q_{accept}$, is labelled with a special name ``Accept."
\item If the machine is in an $\exists$ state, and there is an edge from $C$ to $C'$, and $C'$ is labelled ``Accept," then we label $C$ ``Accept."
\item Similarly for $C$ and any reachable configuration $C'$ from $C$, we label $C$ ``Accept."
\item We say $M$ accepts $x$ if, at the end of recursively applying rules, $C_{start}$ is labelled ``Accept."
\end{enumerate}

So how powerful is $\AP$? It seems that it can be very powerful! However, we show that it can only have polynomial space:
\begin{theorem}
$\AP = \PSPACE$.
\end{theorem}

\begin{proof}
For the $\TQBF$ problem, we can guess values for each $\exists$ variable using an $\exists$ state, and for each $\forall$ variable using a $\forall$ state. The poly-time computation is done at the very end. Therefore, $\TQBF \in \AP$, and since $\TQBF$ is $\PSPACE$-complete, we have $\PSPACE \subseteq \AP$.

\par For the other direction, we do a similar proof to $\NP \subseteq \PSPACE$. Let $L \in \AP$. Then, a ``game" (by making a game tree) is to see whether $x \in L$ has poly depth in the game tree. If the current node is a leaf (i.e., halting configuration), output the current value. Otherwise, recursively observe sub-game trees corresponding to the current node. 

\par We define two players $P_2$ and $P_2$. If the state is $\exists$, output that $P_1$ is the winner if and only if $P_1$ wins at least one of the sub-game trees. Otherwise, if the state is $\forall$, output that $P_2$ is the winner if and only if $P_2$ wins at least one of the sub-game trees. 

\par The algorithm is correct - it also runs in poly space because it only has poly depth, and each recursion we only need to store the current node's name, which is poly in length. Therefore, $L \in \PSPACE$.
\end{proof}

\subsection{Exercises}
\begin{enumerate}

\item Show that if $\NP = \coNP$, then $\PH = \NP$. Generalize this and show that if $\Sigma_k^\P = \Pi_k^\P$, then $\PH = \Sigma_k^\P$.

\item We defined the polynomial hierarchy in terms of $\P$ - what about $\PSPACE$? Look at $\Sigma_k^\PSPACE, \Pi_k^\PSPACE$ and determine that $\PSPACE = \bigcup_{k \ge 1} \Sigma_k^\PSPACE = \bigcup_{k \ge 1} \Pi_k^\PSPACE$.
\end{enumerate}
\section{Boolean Circuits}

We have looked a little at boolean formulas, which take input variables that have values ``true" or ``false" and, through some operations, give a result of either ``true" or ``false" as ``output." Here, we look at circuits, which are a generalization of formulas.

\begin{definition}
A \emph{boolean circuit} is a DAG (directed acyclic graph) with a number of ``source" vertices (those with no incoming edges), also called ``gates," and a ``sink" vertex (with no outgoing edges), also called the ``output." Vertices are labelled with $\wedge, \vee, \neg$ - those with $\wedge, \vee$ have fan-in 2 and fan-out 1, and those with $\neg$ have fan-in and fan-out 1. The \emph{size} of a circuit is the number of vertices it has. The output of a particular input $x \in \{0, 1\}^n$ is applying each of the vertex ``rules" recursively until the output is reached. 
\end{definition}

\begin{theorem}
Any circuit with its $\wedge, \vee$ vertices having bounded fan-in (i.e., $\ge 3$) can be converted to an equivalent circuit with these vertices having only fan-in 2. 
\end{theorem}

\begin{definition}
A \emph{$T(n)$-size circuit family} is a sequence of circuits (i.e., $\{C_n\}_{n \in \mathbb{N}}$), where, for a given function $T \colon \mathbb{N} \rightarrow \mathbb{N}$, has that all of the circuits have size $\le T(n)$ for all $n$. We say that $L \in \SIZE(T(n))$ if there exists a $T(n)$-size circuit family such that for all $x \in \{0, 1\}^n$, $x \in L$ if and only if $C_n(x) = 1$ (i.e., ``accepts" $x$). 
\end{definition}

We start with an example: $L_1 = \{1^n \colon n \in \mathbb{Z}\}$.
\begin{theorem}
\label{thm:unary_linear_circuit_family}
$L_1$ can be decided by a linear-sized circuit family. 
\end{theorem}

\begin{proof}
The circuit is simply a tree of $\wedge$ gates that computes that of all inputs. 
\end{proof}

\begin{definition}
We define $\Ppoly$ to be $\bigcup_{c} \SIZE(n^c)$; in other words, $\Ppoly$ is the set of languages with poly-size circuit families.
\end{definition}

\begin{theorem}
$\P \subseteq \Ppoly$, and the inclusion is strict.
\end{theorem}

\Comment{Add proof}

\begin{theorem}
All unary languages are in $\Ppoly$.
\end{theorem}

\begin{proof}
Implied by \Cref{thm:unary_linear_circuit_family}.
\end{proof}

\subsection{\lang{CKT-SAT}}
As we proved the Cook-Levin Theorem from before, now we prove the circuit analog of $\SAT$. 
\section{Randomization}

In the previous sections, we have only dealt with (non)-deterministic TMs. However, there are many important classes that relate a TM when using a probability, as we define below:

\begin{definition}
A \emph{probabilistic TM} (PTM) has two transition functions, and at each step we choose, independently with probability $\frac{1}{2}$, to apply one of the two functions, like a coin flip. We say that a PTM $M$ halts in $T(n)$ time if $M$ halts on all inputs $x$ within $T(|x|)$ steps, regardless of the choices made during execution.
\end{definition}

We can now define the probabilistic analog of $\P$, which is $\BPP$:
\begin{definition}
A language $L$ is in $\BPTIME(T(n))$ if there exists a PTM $M$ that decides $L$ in time $T(n)$, $M$ halts within $T(|x|)$ steps regardless of choices, and $\Pr[M(x) = L(x)] \ge \frac{2}{3}$. We define:
\[
\BPP = \bigcup_{c \ge 0}\BPTIME(n^c)
\]
\end{definition}

We also give an alternate definition for $\BPP$:
\begin{definition}
$L \in \BPP$ if there exists a poly-time TM verifier $M$ and a polynomial $p$ such that for all $x \in \{0, 1\}^*$, $\Pr[L(M) = L] \ge \frac{2}{3}$ where $M$ has certificates of size $p(|x|)$.
\end{definition}

\begin{theorem}
$\BPP \subseteq \Ppoly$.
\end{theorem}

This proof will use what is called as ``success amplification" - the essential idea is that, given we have polynomial time, we can make successive coin flips to reduce the error bound for $\BPP$ to even less error. We do this as follows:
\begin{itemize}
\item For a PTM $M$ that computes a language $L$ with error $\frac{1}{2} - \frac{1}{n^c}$, run $M$ for $k$ independent trials, and accept only if at least half of the trials accept. 
\item Define $X_i = 1$ if the $i$-th trial accepts, and 0 otherwise, and $X = \Sigma_1^k X_i$. 
\item Therefore, $\Pr[X_i = 1] \ge \frac{1}{2} + \frac{1}{n^c}$. Call this result $p$.
\item $\E[X] \ge pk$ for the same reasoning.
\item Call $\delta = 1-\frac{1}{2p}$.
\end{itemize}
Therefore, now we want to find the probability that the trials done above are incorrect, by using Chernoff bounds:
\begin{center}
$\Pr[X < (1-\delta)pk] \le \exp(\frac{-{(1-\frac{1}{2p})^2}pk}{2}) \le \exp(-\frac{k}{2n^{2c}})$
\end{center}
So, for $k$ a polynomial (i.e., $2n^{2c+d}$), we can have an exponentially small error bound. Now, onto the proof.

\begin{proof}
Let $L \in \BPP$. By success amplification, there is a poly-time TM $M$ such that $L(M) = L$ with error smaller than $2^{-n}$. Choose an $x \in \{0, 1\}^n$. Therefore, $\Pr[\text{$M$ computes incorrectly on $x$}] < \frac{1}{2^n}$. We can observe that:
\[
\Pr[\text{$M$ computes incorrectly on some $y \in \{0, 1\}^n$}] \le \Sigma_{x \in \{0, 1\}^n} \Pr[\text{$M$ computes incorrectly on $x$}] = 1
\]
Therefore, there are some coin flips where $M$ does not make any error on any $x \in \{0, 1\}^n$. All we need to do is convert $M$ into a circuit and hardwire these coin flips, which completes the proof. 
\end{proof}

\begin{theorem}[Sipser-G\'{a}cs]
$\BPP \subseteq \Sigma_2^\P \cap \Pi_2^\P$.
\end{theorem}

\begin{proof}
As before, assume that for an $L \in \BPP$ there is a poly-time TM $M$ such that $L(M) = L$ with error smaller than $2^{-n}$. Since $\BPP$ is closed under complement, we only need to show $\BPP \subseteq \Sigma_2^\P$.

\par For $x \in \{0, 1\}^n$, let $S_x$ be the set of certificates for which $M$ accepts $x$, and let $p$ be the polynomial that is the number of random bits $M$ uses. Therefore, either $|S_x| \ge (1-2^{-n})2^{p(n)}$, or $|S_x| \le 2^{p(n)-n}$, whether $x \in L$ or not. What we will show is that we can differentiate between these using only 2 quantifiers.

\par For $A \subseteq \{0, 1\}^{p(n)}$ and $b \in \{0, 1\}^{p(n)}$, define $A+b = \{a+b \colon a \in A\}$, where the operation is done $(\mod 2)$. Let $k = \lceil {p(n)/n} \rceil +1$. We will show the following:

\begin{enumerate}
\item For all $S \subseteq \{0, 1\}^{p(n)}$ and $|S| \le 2^{p(n)-n}$ and any $k$ vectors $\{u_i\}_{1 \le i \le k}$ with $u_i \in \{0, 1\}^{p(n)}$, we have that $\bigcup_{i=1}^{k} (S+u_i) \ne \{0, 1\}^{p(n)}$.

\item For all $S \subseteq \{0, 1\}^{p(n)}$ and $|S| \ge (1-2^{-n})2^{p(n)}$, there exist $k$ vectors $\{u_i\}_{1 \le i \le k}$ with $\bigcup_{i=1}^{k}(S+u_i) = \{0, 1\}^{p(n)}$.
\end{enumerate}
In other words, we can establish that $x \in L$ with only 2 quantifiers. Proving these implies that $\BPP \subseteq \Sigma_2^\P \cap \Pi_2^\P$. For the first proof, we have that $|S + u_i| = |S|$, so the union is at most $k|S| < 2^{p(n)}$ for large enough $n$.

\par For the second proof, if the $u_i$ are chosen independently randomly, then $\Pr[\bigcup_{i=1}^{k}(S+u_i) = \{0, 1\}] > 0$ will imply the proof. Let $B_r$ be the event that $r \notin \bigcup_{i=1}^{k}(S+u_i)$. This implies that we need to prove that $\Pr[\exists B_r] < 1$ for some $r \in \{0, 1\}^{p(n)}$. Equivalently, we can show that $\Pr[B_r] < 2^{-p(n)}$ for all $r$. 

\par Define $B_r^i$ the event that $r \notin S + u_i$ (equivalently, $r + u_i \notin S$). We can see that $B_r = \bigcap_{1 \le i \le k}B_{r}^i$. However, $r+u_i$ is uniformly chosen from $\{0, 1\}^{p(n)}$, so $r+u_i \in S$ with probability $\ge 1-2^{-n}$. Also, since the $B_r^i$ are independent for all $i$, we have $\Pr[B_r] = (\Pr[B_r^i])^k \le 2^{-nk} < 2^{-p(n)}$. 
\end{proof}

\subsection{$\RP, \coRP, \ZPP$}

\subsection*{Exercises}
\newcommand{\BPPpoly}{\BPP/\poly}
\begin{enumerate}
\item Show that $\BPPpoly = \Ppoly$. % http://zoo.cs.yale.edu/classes/cs468/fall12/468Solutions2.pdf
\item Show that if $\NP \subseteq \BPP$, then $\NP = \RP$. % http://www.inf.ed.ac.uk/teaching/courses/cmc/cw3_solns.pdf
\end{enumerate}
\section{Interactive Proofs}

We have talked before about $\NP$ and how we needed to provide a ``certificate" to prove that it truly is in the language (or is not, given a bad certificate). But what if we have a verifier and a prover that interact with each other? We can actually have more power for some canonical problems. This is called an ``interactive proof."

\begin{example}
Let's observe an interactive proof for $\phi \in \ThreeSAT$. For each clause in $\phi$, one at a time, have the prover give the values of the literals in the clause to the verifier, and have the verifier keep the answers until the end. At any point, if the prover gives conflicting values for the literals, reject. Otherwise, have the verifier check with the given values if all clauses are satisfied.
\end{example}

Now the last example works over each clause, so there are $n$ ``rounds" of interaction between the prover and verifier. Now we make this explicit:

\newcommand{\out}[2]{\texttt{out}$_{#1}\langle {#1}, {#2} \rangle(x)$}
\begin{definition}
A \emph{k-round} interaction of two functions $f, g$, with $k$ a non-negative integer (not necessarily independent of $f$ and $g$) is a sequence of strings $\{a_i\}_{i \ge 1}$:
\begin{itemize}
\item $a_1 = f(x)$
\item $a_2 = g(x, a_1)$
\item $\cdots$
\item $a_{2m+1} = f(x, a_1, \cdots, a_{2m})$, if $2m < k$
\item $a_{2m+2} = g(x, a_1, \cdots, a_{2m+1})$, if $2m+1 < k$.
\end{itemize}
We define the \emph{output} of $f$ (or $g$) to be $f(x, a_1, \cdots, a_k)$ at the end of the interaction (a common notation of this is \out{f}{g}). 
\end{definition}

Now we know what an interaction is, we need to know what a ``proof system" is:

\newcommand{\dIP}{\lang{dIP}}
\begin{definition}
$L$ is a \emph{k-round deterministic interactive proof system} if there is a $\poly(|x|)$-time TM $V$ that on input $x, a_1, \cdots, a_i$, has a $k$-round interaction with a function $f$ such that:
\begin{itemize}
\item If $x \in L$, then there is a function $f$ such that \out{V}{f} accepts.
\item If $x \notin L$, then for all functions $f$ \out{V}{f} rejects.
\end{itemize} 
Define $\dIP$ to be the set of languages with a $\poly(n)$-round deterministic interactive proof system. 
\end{definition}

\begin{theorem}
$\dIP = \NP$.
\end{theorem}

\begin{proof}
$\NP \subseteq \dIP$ because every $\NP$ language has a 1-round DPS. Suppose $L \in \dIP$, and $V$ is the verifier. Then, a certificate that $x \in L$ is a set of strings $\{a_i\}_{1 \le i \le k}$ that has $V$ accept. For verification, the verifier needs to check if $V(x) = a_1, V(x, a_1, a_2) = a_3$, and so on and that $V(x, a_1, \cdots, a_k)$ accepts. However, if the set of strings exists, then we can use a prover function $P$ such that $P(x, a_1) = a_2, P(x, a_1, a_2, a_3) = a_4$, and so on. This shows that \out{V}{P} accepts, meaning that $x \in L$. 
\end{proof}
\section{Quantum Computation}

All of what we have studied is the ``classical" model of computation; basically, when we store and retrieve values, we know that we will exactly get what is there. In this lecture, we will diverge to a different model of computation: quantum computing.

\par In a sense, quantum computing allows us to perform many calculations at the same time without actually performing them. It is open whether or not, in actuality, quantum computing is faster than classical computing. 

\par One can think of classical TMs - as before, we have represented the entire state of the machine as a ``snapshot" or a ``configuration." We can also call this the current ``state" of the TM. We can abstract even further: take all possible states of the TM, and write that as a vector. The current state has value 1 and all others are 0. However, this is a very inefficient way of viewing its computation.

\par We will study a complexity class $\BQP$, standing for ``Bounded error, Quantum, Polynomial Time." Basically, $\BQP$ contains all problems that are efficiently computable with high probability on a quantum computer. 

\newcommand{\ket}[1]{\ensuremath{\vert #1 \rangle}}

\par In a quantum computer, we also have states, but they are more probabilistic in nature. We consider the state of a quantum computer to be: $\sum_{i=1}^{N}\alpha_i \ket{i}$, where the $\ket{i}$ are basis vectors, $\alpha_i \in \mathbb{C}$, and $\sum_{i} |\alpha_i|^2 = 1$. We also call the $\alpha_i$ ``qubits" (stands for ``quantum bits").

\par We usually work over a ``Hilbert space" when doing quantum computing. To do so, we need to define it first:
\begin{definition}
A Hilbert space is a vector space over $\mathbb{C}$, with vectors denoted $\ket{\psi}$, such that:
\begin{itemize}
\item It has an inner product $\langle a \vert b \rangle$ (maps vectors to $\mathbb{C}$ with the properties that:
\begin{itemize}
\item $\langle \psi \vert \psi \rangle$ is positive if $\ket{\psi}$ is not the 0-vector (positivity).
\item $\langle \psi \vert (a \vert \psi_1 \rangle + b \vert \psi_2 \rangle) \rangle = a\langle \psi \vert \psi_1 \rangle + b \langle \psi \vert \psi_2 \rangle$ (linearity in the field).
\item $\langle a \vert b \rangle = \langle b \vert a \rangle^*$, where $^*$ means complex conjugation (skew-symmetric).
\end{itemize}
\item $||\psi|| = \langle \psi \vert \psi \rangle^{\frac{1}{2}}$ (``complete").
\end{itemize}
\end{definition}

We call a ``quantum register" a set of $m$ qubits, which is just a linear equation over all $2^m$ possible states. Say that the vector of the $\alpha_i$ here is $v \in \mathbb{C}^{2^m}$: we have $\sum_{x}|v_x|^2 = 1$. 

\par Now that we know what qubits and registers are, how do we work with them? We use a ``quantum operation" to do so:
\begin{definition}
A quantum operation is a function $f \colon \mathbb{C}^{2^m} \rightarrow \mathbb{C}^{2^m}$ that maps the previous state to the new state, is linear, and preserves norms (i.e., unit vectors to unit vectors).
\end{definition}

\begin{example}
A common operation in classical computing is ``flipping" an $m$-bit register - swap the 0's and 1's. How do we do this in a quantum computer? For $b \in \{0, 1\}, x \in \{0, 1\}^{m-1}$, we map $\ket{b, x}$ to $\ket{1-b, x}$. 
\end{example}

\section{PCP Theorem}

One of the reasons we study $\NP$ and $\NP$-completeness is that we can verify membership in a language by a poly-size ``witness" or ``certificate" or ``advice" (in the case of $\Ppoly$). In the case of $\SAT$, the certificate would be the set of assignments to the $x_i$ variables. However, if we are given a very large instance, then poly-size can be ``too large," in a sense. It would be very great if we can still verify membership in $\NP$ (or other classes) without having to give the entire certificate.

\par The $\PCP$ theorem will help us do this. It allows any mathematical proof (including the special case of certificates) to be transformed in a way to make them ``checkable" while only querying a few of the bits, and accepts the result with high probability in many cases. We show using the $\PCP$ theorem that for any $\NP$-complete optimization problem, even approximating the optimal solution is just as difficult as computing the exact one (unless $\P = \NP$, of course).

\begin{definition}
Define a \emph{nonadaptive verifier} to be one that selects queries only based on its input and random tape (i.e., does not rely on past queries). Let $q, r$ be functions from $\mathbb{N} \rightarrow \mathbb{N}$. We have that a language $L$ has an $(r(n), q(n))-\PCP$ verifier if there is a poly-time PTM $M$ (equivalently, a probabilistic algorithm) with the following properties:
\begin{itemize}
\item On input $x$ with $n = |x|$, and given access to a random string $u$ with $|u| \le q(n)^{r(n)}$ (the ``proof"), $M$ uses at most $r(n)$ random flips of a coin and at most $q(n)$ nonadaptive queries to locations in $u$ (with accept/reject in the usual sense). Let $M^u(x)$ be $M$'s output on $x$ with random access to $u$. 
\item If $x \in L$, there is a proof $u$ (the ``correct proof") such that $\Pr[M^u(x) = 1] = 1$.
\item If $x \notin L$, then for all proofs $u$, $\Pr[M^u(x) = 1] \le \frac{1}{2}$. 
\end{itemize}
We have that $L \in \PCP(r(n), q(n))$ if $L$ has a $(O(r(n)), O(q(n)))-\PCP$ verifier (we sometimes use the constants in the $O()$ notation). 
\end{definition}

\begin{theorem}[$\PCP$ theorem]
$\NP = \PCP(\log n, 1)$.
\end{theorem}

Another ``scaled-up" version of the $\PCP$ theorem is:
\begin{theorem}
$\PCP(p(n), 1) = \NEXP$ for a polynomial $p$.
\end{theorem}

\subsection{Hardness of Approximation}
\newcommand{\MAXThreeSAT}{\lang{MAX-3SAT}}
We won't cover the ingredients for proving the $\PCP$ theorem in this lecture. However, we will see some applications of the theorem as well as equivalent formulations. We define $\MAXThreeSAT$ as an optimization version of the $\ThreeSAT$ problem:

\begin{definition}
Define $\texttt{val}(\phi)$ as the maximum possible fraction of clauses that can be satisfied (i.e., satisfiable if and only if $\texttt{val}(\phi) = 1$).
\end{definition}

Therefore we have the equivalent definition of the $\PCP$ theorem:
\begin{theorem}
There exists a constant $\rho < 1$ such that for all $L \in \NP$ there is a poly-time function $f$ such that if $x \in L$, then $\texttt{val}(f(x)) = 1$, and if $x \notin L$, then $\texttt{val}f(x)) < \rho$.
\end{theorem}
\section{Decision Trees}

Since most questions about TMs turn out to be undecidable, we should shift our attention to a more limited model of computation. We now look at decision trees, which involve boolean functions. 

\begin{definition}
Let $f$ be a boolean function. A \emph{decision tree} of $f$ is a tree for which each non-input, non-output (``leaf") vertex has a label $v_i$, and two outgoing edges with labels 0, 1.
\end{definition}

Suppose we are given input $x_1x_2\cdots x_n$. If $x_1 = 1$, we proceed down the `1' path of $v_1$, and then recursively call the next vertex on input $x_2\cdots x_n$ until either we run out of input or we reach a leaf node (and may possibly have some input left). 

\begin{definition}
The \emph{cost} of a decision tree $T$ on input $x$ is the number of bits examined by $x$ by $T$, denoted by $cost(T, x)$. The \emph{decision tree complexity} (or just ``complexity") of a function $f$, $D(f)$ is $\min_{T \in T_f} \max_{x} cost(T, x)$, where $T_f$ is the set of all decision trees that compute $f$. 
\end{definition}

We note that $D(f) \le n$ for all $f$ because we can just make a full binary tree with $2^n$ vertices. We want to know when this bound is the only one known, or if there is a smaller upper bound.

\begin{example}
The ``or" function with $f(x_1, \cdots, x_n) = \bigvee_{i=1}^{n}x_i$ has no better bound than $n$ due to an adversarial argument. 
\end{example}

Like for other models of computation, we often introduced a ``nondeterministic" version of that model. Here we do so with certificate complexity:
\begin{definition}
Let $f$ be a function and $x$ an input. A \emph{0-certificate} of $x$ is a subset of indices $S$ such that $f(x') = 0$ for all $x'$ such that $x'_{S} = x_{S}$, where $x_{S}$ is the substring of $x$ with indices in $S$. We define \emph{1-certificate} similarly. The \emph{certificate complexity} of $f$, $C(f)$, is the minimum $k$ such that every $x$ has a $f(x)$-certificate of size at most $k$. 
\end{definition}

We have that $C(f) \le D(f)$, but sometimes the inequality is strict. But how much less can $C(f)$ get? It turns out that it cannot get too small:

\begin{theorem}
$D(f) \le C(f)^2$ for all $f$.
\end{theorem}

\begin{proof}
Let $k = C(f)$, and for all $x$, let $S_x$ be the $k$-sized subset of $\{1, \cdots, n\}$ which is the $f(x)$-certificate for $x$. Set $X = \{0, 1\}^n$. If there is a $b \in \{0, 1\}$ such that $f(x) = b$ for all $x \in X$, then output $b$. Otherwise, choose any $y \in X$ such that $f(y) = 0$ and query everything in $S_y$ not previously queried. Remove from X every $z \in \{0, 1\}$ not consistent with the answers. Repeat until $f$ has the same value on all remaining strings in $X$. 

\par Since all $x$ have some certificate that proves $f(x)$ correctly, this algorithm terminates. Each time it queries bits in a 0-certificate, all 1-certificates decrease by at least 1 because each 1-certificate intersects each 0-certificate; if this were not true, then there is a single input that contains both certificates, which cannot happen. Therefore, in $k$ iterations, all 1-certificates decrease to 0, which means everything in $X$ only has 0-certificates, and the algorithm answers 0. Since each iteration only queries $k$ bits, the algorithm finishes after $k^2$ queries.
\end{proof}
\include{\dir/08_communication}
\section{Algebraic Computation Models}

In computational models such as deterministic TMs, we look at a computational step as an application of a transition function, and then repeating until some end condition (or runs forever). Now we look at ``algebraic" computational models, which generalize classical computational models like $\P, \NP$ to mathematical fields, say the reals or complex numbers. 

\begin{definition}
Let $\mathbb{F}$ be a field. We call an \emph{algebraic straight program of length $T$} with inputs $x_i \in \mathbb{F}, 1 \le i \le n$, constants $c_i \in \mathbb{F}$ to be a sequence of $T$ statements of the form $a_i = b_{i_1} OP\; b_{i_2}$, where $OP$ is one of the field operations and the $b_{i_j}$ are inputs, constants, or a previous $a_i$. 
\end{definition}

\begin{theorem}
Any straight-line program of length $T$ with $n$ input variables is a polynomial with degree at most $2^T$.
\end{theorem}

\begin{proof}
Each input is of degree 1, and every step either adds or multiplies two polynomials. Since the produce of 2 polynomials has degree at most their sum, we have the result.
\end{proof}
\section{Counting Complexity}

We have defined the $\P$ class as the set of languages that are solvable on a deterministic TM in polynomial time. However, we did not actually describe the solution(s) - for example, we only described whether a given 3SAT formula is satisfiable, and not how many satisfying solutions it has. Note that describing solutions is not a decision problem. However, we will make the necessary modifications:

\newcommand{\SharpP}{\lang{\#P}}
\begin{definition}
A function $f \in \SharpP$ if there is a polynomial $p$ and a poly-time TM $M$ such that for all $x$, $f(x) = |\left \{y \in \{0, 1\}^{p(|x|)} \colon M(x, y) = 1\right \}|$.
\end{definition}

We can see that instead of determining membership, $f$ counts the number of solutions such that the TM accepts. Clearly, the problems in $\SharpP$ are ``harder" than those in $\P$. But what if we want to have a decision version on $\SharpP$? We can do so with $\PP$:

\begin{definition}
We have $L \in \PP$ if there is a poly-time TM $M$ and a polynomial $p$ such that for all $x$, $x \in L$ if and only if $|\left\{u \in \{0, 1\}^{p(|x|)} \colon M(x, u) = 1\right\} | \ge 2^{p(|x|)-1}$.
\end{definition}
We can see that $\PP$ is basically a ``majority selector," in that $x \in L$ if more than half of all certificates have the TM accept. For the next theorem, we define $\FP$ as the set of functions computable by a deterministic poly-time TM.

\begin{theorem}
$\PP = \P$ if and only if $\SharpP = \FP$. 
\end{theorem}

\begin{proof}
It is easy to see that $\SharpP = \FP$ implies $\PP = \P$: given input $x$, just count the number of $u$ such that $M(x, u) = 1$, and compare. For the other direction
\end{proof}
\Comment{Complete this proof}

\subsection{$\SharpP$-completeness}
\begin{definition}
Define $\FP^{f}$ for a function $f$ to be the set of functions computable by poly-time TMs with oracle access to $f$ (same notation as for problems). We have $f$ to be $\SharpP$-complete if:
\begin{itemize}
\item $f \in \SharpP$
\item For all $g \in \SharpP$, $g \in \FP^{f}$.
\end{itemize}
\end{definition}

\newcommand{\SharpSAT}{\lang{\#\SAT}}
\begin{theorem}
$\SharpSAT$ is $\SharpP$-complete.
\end{theorem}

\begin{proof}
The reasoning is that the reduction for $\SAT$ is \emph{parsimonious} (i.e., is bijective and preserves counts). 
\end{proof}

\newcommand{\SharpPSPACE}{\lang{\#\PSPACE}}
\subsection{$\SharpPSPACE$}
We have talked about $\SharpP$ - what about $\SharpPSPACE$? We define it below:
\begin{definition}
A function $f \in \SharpPSPACE$ if there is a poly-time TM $M$ such that $f(x)$ outputs the number of possible $x_1, \cdots, x_k$ such that $Q_1x_1 \cdots Q_kx_k [M(x_1, \cdots, x_k) = 1]$. 
\end{definition}
\newcommand{\SharpQBF}{\lang{\#QBF}}
We also define $\SharpPSPACE$-completeness the same way we have $\SharpP$-completeness. We can easily see that $\SharpQBF \in \SharpPSPACE$, where $\SharpQBF$ is the same as the $\lang{QBF}$ problem but we count the number of solutions. We will show that $\SharpQBF = \SharpSAT$, thereby giving the result that:
\begin{theorem}
$\SharpP = \SharpPSPACE$.
\end{theorem}

\begin{proof}
The way we will prove this is showing $\SharpQBF$ is $\SharpPSPACE$-complete, and $\SharpQBF = \SharpSAT$.

We clearly have $\SharpQBF \in \SharpPSPACE$ (there is a poly-space TM $M$ that for any boolean formula $F$, and quantification of variables, $M$ accepts on the quantification if and only if the quantification for the formula is true).

\par Let $f \in \SharpPSPACE$. There is a poly-space TM $M$ and polynomial $p$ such that $f(x)$ is the number of $x_1, \cdots x_k$ has $Q_1x_1\cdots Q_{p(|x|)}x_{p(|x|)} M(x, x_1, \cdots, x_{p(|x|)})$ accepts.

\par Let $F_x(x_1, \cdots x_{p(|x|)})$ be the equivalent formula of $M$ (this is of poly-size). Therefore, $f(x)$ is the number of quantifications such that $F_x$ is true, which is equal to counting the solutions of the formula (i.e., $\SharpQBF(F_x)$). Therefore, $f \le_{\SharpP} \SharpQBF$. 

\par Now we show $\SharpQBF = \SharpSAT$. Let $F$ be a boolean formula on $n$ variables. We prove by induction on $n$. For $n=0$, the two formulas with no variables are the true formula $t$ and the false one $f$. $\SharpSAT(t) = \SharpQBF(t) = 1$ (there are no quantifications), and $\SharpSAT(f) = \SharpQBF(f) = 0$. 

\par We assume that $\SharpSAT(m) = \SharpQBF(m)$ for all formulas $m$ with $n-1$ variables. Let $F'$ be a formula on $n$ variables, and let $F'_t, F'_f$ be the same formula but with the last variable made true and false, respectively. We note the following:
\begin{itemize}
\item $\SharpSAT(F'_t \vee F'_f) = \SharpSAT(F'_t) + \SharpSAT(F'_f) - \SharpSAT(F'_t \wedge F'_f)$.
\item Let $Q^kx = Q_1x_1 \cdots Q_kx_k$ be the first $k$ quantified variables of $F'$. Then $Q^{n-1}x\forall x_n F'$ is true if and only if $Q^{n-1}x(F'_t \wedge F'_f)$ is true, and $Q^{n-1}x\exists x_n F'$ is true if and only if $Q^{n-1}x(F'_t \vee F'_f)$ is true. 
\end{itemize}
We also see that $\SharpQBF(F')$ is the number of quantifications of $F'$ that make it true with the $n$th variable as $\exists$ + the same but the $n$th variable is $\forall$. Therefore, $\SharpQBF(F') = \SharpQBF(F'_t \wedge F'_f) + \SharpQBF(F'_t \vee F'_f)$.

\par We assume by the induction hypothesis that this is equivalent to $\SharpSAT(F'_t \wedge F'_f) + \SharpSAT(F'_t \vee F'_f)$. Also, by the first observation, this is equivalent to $\SharpSAT(F'_t) + \SharpSAT(F'_f)$, which is equal to $\SharpSAT(F')$. 

\par We have $\SharpP = \SharpPSPACE$ because of our definition of $\SharpP$-completeness. 

%We introduce the ``circuit value problem" (CVP), which is: given a boolean formula $\phi$ and an assignment $x$ to its variables, is $\phi$ true with this assignment? It is much easier than solving $\SAT$, and is in $\P$. 
%
%\par We show that $\SharpQBF \in \SharpPSPACE$: let $M$ be a TM that decides CVP, and $Q_1x_1 \cdots Q_kx_k [\phi]$ be an arbitrary QBF. Therefore, the number of settings to the $x_i$ such that $M(\phi, x)$ accepts is precisely the number of $x_i$ settings that make $\phi$ true. Therefore, $\SharpQBF \in \SharpPSPACE$.
%
%\par Let $f \in \SharpPSPACE$. Therefore, there is a poly-time TM $M$ such that $f(x)$ is the number of $Q_ix_i$ settings that $M(x) = 1$. Let $L$ be the set of $x$ such that $M(x) = 1$. We can see that $L \in \P$ (just run $M$ on the input). Let $g$ be the poly-time reduction from $L$ to CVP (i.e., $M(x) = 1$ if and only if $g(x)$ is a formula with a satisfying truth assignment). 
%
%\par Therefore, $f(x)$ is the number of $Q_ig(x_i)$ settings that has the 
\end{proof}

\subsection{Toda's Theorem}
How powerful is $\SharpP$? It has been hard to determine until 1989, with one amazing result by Toda:
\begin{theorem}
$\PH \subseteq \P^{\#\SAT}$. 
\end{theorem}
This means that with any $\SharpP$-complete problem, we can solve any problem in $\PH$ (i.e., a subexponential algorithm). We build the necessary ingredients for the proof:

\newcommand{\Parity}{\oplus}
\newcommand{\ParityP}{\lang{\Parity\P}}
\begin{definition}
We have $\L \in \ParityP$ (``parity-$\P$") if there is a poly-time \emph{nondeterministic} TM $M$ such that $x \in L$ if and only if there are an odd number of accepting paths of $M$ on $x$. 
\end{definition}
\newcommand{\ParitySAT}{\Parity\SAT}
Also, we need to know about the $\Parity$ quantifier as well as $\ParitySAT$:
\begin{definition}
For a boolean formula $\phi$, define $\Parity\phi(x)$ to be true if the number of $x$'s that make $\phi$ true is odd. Define $\ParitySAT$ to be the language of $\Parity\phi(x)$ where $\phi$ is an unquantified boolean formula in any form.
\end{definition}
We can reason that $\ParityP$ corresponds to the least significant bit of problems in $\SharpP$, so it doesn't seem very powerful. However, in the first part of the proof, Toda shows a randomized reduction from $\PH$ to $\ParitySAT$. 
\Comment{Add other parts of proof}

\include{\dir/11_averagecase}
\include{\dir/12_hardnessamplification}
\include{\dir/13_derandomization}
\include{\dir/14_expanders_extractors}
\section{PCP and Fourier Transform}
\section{Parameterized Complexity}

We know of $\P$ and $\NP$, and that there are many problems in each (and complete ones in $\NP$). However, we don't have any granularity when it comes to separating the two classes. A relatively recent technique is parameterization, which we discuss here. The goal is to find parameters of an input that make a particular language (i.e., problem) hard. 

\begin{definition}
A \emph{parameterized problem} $P$ is a set of tuples $(w, k)$ where $w$ is the \emph{input} and $k$ is the \emph{parameter} (the alphabets to which $w$ and $k$ belong are disjoint). We now define the pair of a language and the parameter:
\[
(L, k) = \{(w, m) : w \in L, m = k(w)\}.
\]
\end{definition}

We want to know whether $(w, k) \in P$ is true or not. The notion of a ``parameter" is basically a function that takes instances of a problem, and returns a non-negative integer. One possible parameter for vertex cover is: does there exist a vertex cover of size $\le k$?

\begin{definition}
We define the following classes (``solvable" means on a deterministic TM):
\begin{center}
$\XP = \{(L, k) : (L, k)$, for a given $w \in L$, is solvable in $|w|^{f(k)}$ time\}.\\
$\FPT = \{(L, k) : (L, k)$, for a given $w \in L$, is solvable in $|w|^{O(1)} \times f(k)$\}.
\end{center}
\end{definition}

Note that there is no bound on the growth of $f$; the only thing we must guarantee is that the running time of $M$ is polynomial in the length of $w$, if we assume a fixed parameter. This is why we usually do not study $\XP$, because the polynomial changes as $k$ changes (also, most languages are in $\XP$ anyway). Therefore, we study $\FPT$ by looking at the following problem:

\begin{theorem}
Define k-Vertex Cover $(KVC, k)$= $\{(G, k) : G = (V, E)$ is a graph and contains a vertex cover of size $k$\}. We have that $(KVC, k) \in \XP$, and $(KVC, k) \in \FPT$.
\end{theorem}

\begin{proof}
\par The parameter here is $k \le |V|$. We have that it is trivially in $\XP$, because we can check all possible $|V|^k$ subsets of the vertices, and each requires linear time in the number of vertices. Therefore, we have an $O(|V|^{k+1})$ algorithm (which has $f(k) = k+1$ here).

\par For showing it is in $\FPT$, we work as follows:
\begin{enumerate}
\item Choose an edge, and include one of the end vertices of the edge in the vertex cover. 
\item Delete all edges incident to the chosen vertex.
\item Repeat $k$ times.
\item Each repetition, check if there is a vertex cover; if not, repeat again. 
\end{enumerate}

In this algorithm, we end up checking $2^k$ possible different choices of vertices. Each step takes $O(|V|)$ time, so we have an $O(|V| \cdot 2^k)$ algorithm (having $f(k) = 2^k$), which shows that it is in $\FPT$.
\end{proof}

\bibliographystyle{alpha}
\bibliography{lecture_notes}

\end{document}